\documentclass[11pt,a4paper]{article}
%\usepackage[utf8]{inputenc}
\usepackage{amsthm}
\usepackage{amsfonts}
\usepackage{amsmath}
\usepackage{amssymb}
\let\amssquare\square
\usepackage{mathtools}
%\usepackage{stmaryrd}
%\usepackage{tensor}
\usepackage[mathscr]{eucal}
\usepackage{url}
%\usepackage{marvosym}
\usepackage[left=2cm,right=2cm,top=2cm,bottom=2cm]{geometry}

\usepackage{tikz-cd}
\usetikzlibrary{cd}
\usetikzlibrary{arrows}

\theoremstyle{plain}
\newtheorem{theorem}{Theorem}[section]
%\newtheorem{axiom}[theorem]{Axiom}
\newtheorem*{theoremstar}{Theorem}
%\newtheorem{fact}[theorem]{Fact}
\newtheorem{proposition}[theorem]{Proposition}
\newtheorem{lemma}[theorem]{Lemma}
\newtheorem{corollary}[theorem]{Corollary}

\theoremstyle{definition}
\newtheorem{definition}[theorem]{Definition}
%\newtheorem{convention}[theorem]{Convention}
%\newtheorem{construction}[theorem]{Construction}
\newtheorem{example}[theorem]{Example}
%\newtheorem{examples}[theorem]{Examples}
%\newtheorem{notation}[theorem]{Notation}
\newtheorem{remark}[theorem]{Remark}
%\newtheorem{idea}[theorem]{Idea}
%\newtheorem{question}[theorem]{Question}



\newcommand{\C}{\mathscr{C}}
\newcommand{\homC}{\underline{\C}}
\newcommand{\D}{\mathscr{D}}
\newcommand{\E}{\mathscr{E}}
\newcommand{\M}{\mathscr{M}}
\newcommand{\N}{\mathscr{N}}

\newcommand{\bN}{\mathbb{N}}
\newcommand{\bZ}{\mathbb{Z}}


\newcommand{\Pastro}{\Phi}
%\newcommand{\Pastro}{\mathrm{Pastro}}
\newcommand{\Double}{\mathcal{D}}

% Categories
\newcommand{\Set}{\mathbf{Set}}
\newcommand{\Cat}{\mathbf{Cat}}
\newcommand{\Prof}{\mathbf{Prof}}
\newcommand{\MonCat}{\mathbf{MonCat}}
\newcommand{\LaxMonCat}{\mathbf{LaxMonCat}}

\newcommand{\Act}{\mathbf{Act}}
\newcommand{\PreOptic}{\mathbf{PreOptic}}
\newcommand{\Optic}{\mathbf{Optic}}
\newcommand{\SemiOptic}{\mathbf{SemiOptic}}
\newcommand{\Lens}{\mathbf{Lens}}
\newcommand{\Prism}{\mathbf{Prism}}
\newcommand{\Hask}{\mathbf{Hask}}
\newcommand{\Endo}{\mathbf{Endo}}
\newcommand{\Strong}{\mathbf{Strong}}
\newcommand{\Tamb}{\mathbf{Tamb}}
\newcommand{\Point}{\mathbf{Point}}
\newcommand{\CoPoint}{\mathbf{CoPoint}}
\newcommand{\Traversable}{\mathbf{Traversable}}

\newcommand{\id}{\mathrm{id}}
\newcommand{\op}{\mathrm{op}}
\newcommand{\const}{\mathrm{const}}
\DeclareMathOperator{\ob}{ob}
\DeclareMathOperator{\copr}{copr}
\newcommand{\inl}{\mathrm{inl}}
\newcommand{\inr}{\mathrm{inr}}
\DeclareMathOperator{\im}{im}

\newcommand{\fget}{\textsc{Get}}
\newcommand{\fput}{\textsc{Put}}
\newcommand{\fmodify}{\textsc{Modify}}
\newcommand{\freview}{\textsc{Review}}
\newcommand{\fmatching}{\textsc{Matching}}

% Special arrows
\newcommand{\isoto}{\xrightarrow{\cong}}
\newcommand{\hto}{\ensuremath{\,\mathaccent\shortmid\rightarrow\,}}

\makeatletter
\providecommand{\leftsquigarrow}{%
  \mathrel{\mathpalette\reflect@squig\relax}%
}
\newcommand{\reflect@squig}[2]{%
  \reflectbox{$\m@th#1\rightsquigarrow$}%
}
\makeatother

% Draft helpers
\newcommand{\todo}[1]{\textcolor{red}{\small #1}}

\newif\ifhideproofs
%\hideproofstrue %uncomment to hide proofs
\ifhideproofs
\usepackage{environ}
\NewEnviron{hide}{}
\let\proof\hide
\let\endproof\endhide
\fi

\title{The Categorical Structure of Profunctor Optics}
\author{Mitchell Riley}
\begin{document}
\maketitle

\section{Introduction}

\todo{TODO: Might have to adjust the entire thing because the action of $\M$ on $\C$ is not strictly monoidal. On the other hand, through the entire Pastro/Street paper pretends associativity/etc is strict, so maybe who cares? What we really want is just a (non-strict) monoidal functor $\M \to [\C, \C]$ from some $\M$. I may also want to put $\otimes$ and $\cdot$ everywhere instead of just writing $MA$ and $NMA$.}

\todo{TODO: Think about size issues, we are taking some very big coends...}

\todo{Past work:
\begin{itemize}
\item The Haskell lens people: edwardk, shachaf, xplat, roconnor... (TODO: find out who these people are)
\item The Doubles paper
\item Bartosz's blog post which was written about the same time I started this. (Only tensor-like optics, no discussion of laws, confusion about enrichment(?))
\item The profunctor optics paper out of Oxford. (No generality, nothing categorical, no discussion of laws)
\end{itemize}
}

\todo{Contributions:
\begin{itemize}
\item ...
\end{itemize}
}

\todo{Introduction should include:
\begin{itemize}
\item Previous attempts to generalise optics
\item Some remark about how it's hard to isolate exactly what $\Set$-like properties are being used
\end{itemize}
}
A lens from $S$ to $A$ is a pair of maps $\fget : S \to A$ and $\fput : S \times A \to S$ that typically obeys three laws (whose names are in diagrammatic order):
\begin{itemize}
\item ``$\fput\fget$'': $\fget \; \fput = \pi_2$ as maps $S \times A \to A$, so that any update to $A$ should be \todo{represented faithfully in $S$},
\item ``$\fget\fput$'': $\fput [\id_S, \fget] = \id_S$ as maps $S \to S$, so that if $A$ is not changed then $S$ should not change; and,
\item ``$\fput\fput$'': $\fput (\fput \times \id_A) = \fput \, \pi_{1, 3}$ as maps $S \times A \times A \to S$, so any update to $A$ completely overwrites previous updates.
\end{itemize}

\todo{talk about ``constant complement'' idea}

\subsection{Notation}
\todo{I am open to changing any/all of these}
\begin{itemize}
\item Given $f : A \to B$ and $g : A \to C$, I write $[f, g] : A \to B \times C$ and $!_A : A \to 1$.
\item Given $f : A \to B$ and $g : C \to D$, I write $f \times g : A \times C \to B \times D$.
\item Given a $x \in P(C, C)$, I write $\langle x \rangle \in \int^C P(C, C)$ for its image, and $\copr_C : P(C,C) \to \int^C P(C, C)$
\item For a monoidal category, the structure maps are $\alpha$, $\lambda$ and $\rho$. A symmetry is $s$.
\end{itemize}

\todo{
\subsection{(Co)ends and (Co)Yoneda}
}

\section{Optics}

\todo{ Notation: For a monoidal category, the structure maps are $\alpha$, $\lambda$ and $\rho$, symmetry is $s$. For an action of $(\M, \otimes, I)$ on $\C$ we have two structure maps that are isomorphisms: $\epsilon_A : A \to IA$ and $\mu_{M,N,A} = MNA \to (M \otimes N)A$. }

\todo{ We elide boring uses of the structure maps so we can stay sane...}

\begin{proposition}
Let $\C$ be a category and $(\M, \otimes, I)$ a monoidal category that acts on $\C$. There is a category $\PreOptic_\M$ that has the same objects as $\C \times \C^\op$, with homsets given by
\begin{align*}
\PreOptic_\M((S, S'), (A, A')) = \int^{M \in \M} \C(S, M A) \times \C(M A', S')
\end{align*}
\end{proposition}
To distinguish these from morphisms in $\C \times \C^\op$, we will write preoptics with a crossed arrow: $p : (S, S') \hto (A, A')$. If $(l,r) : (S \to M A, M A' \to S')$, we write $\langle l, r \rangle : (S, S') \hto (A, A')$ for their image in $\PreOptic_\M((S, S'), (A, A'))$, and say that $M$ is the complement for this representative.
\begin{proof}
In \cite[Section 6]{Doubles} this is proven abstractly, by exhibiting this category as the Kleisli category for a monad in the bicategory $\Prof$. Here we give a direct proof.

The composition map
\begin{align*}
\int^{M \in \M} \left(\C(S, M A) \times \C(M A', S')\right) \times \int^{M \in \M} \left( \C(R, M S) \times \C(M S', R')\right) \to \int^{M \in \M} \C(R, M A) \times \C(M A', R')
\end{align*}
of preoptics is induced by the following morphisms for every $M, N \in \M$:
\begin{align*}
&\C(S, M A) \times \C(M A', S') \times \C(R, N S) \times \C(N S', R')\\
\to \,& \C(NS, NM A) \times \C(NM A', NS') \times \C(R, N S) \times \C(N S', R') && \text{(functoriality of action of $N$)} \\
\to \,& \C(R, N M A) \times \C(N M A', R') && \text{(composition)} \\
\to \,& \C(R, (N \otimes M) A) \times \C((N \otimes M) A', R') && \text{(composition with $\mu_{M,N,A}$)} \\
\to \,&\int^{M \in \M} \C(R, M A) \times \C(M A', R') && \text{($\copr_{N \otimes M}$)}
\end{align*}
In other words, let $\langle l, r \rangle : (R, R') \hto (S, S')$ and $\langle l', r' \rangle : (S, S') \hto (A, A')$ with $M$ the complement for $\langle l, r \rangle$. The composite is then: $\langle l', r' \rangle \circ \langle l, r \rangle = \langle \mu_{M,N,A}(M l')l, r(Mr')\mu^{-1}_{M,N,A'} \rangle$. 

The identity morphism $(S, S') \hto (S, S')$ is given by $\langle \epsilon_S, \epsilon_{S'}^{-1} \rangle$. \todo{which will later be described as $(\id_S, \id_{S'})_*$}

Associativity of composition can be checked equationally. Given $(l_1, r_1) : (R \to MS, MS' \to R')$ and $(l_2, r_2) : (S \to NA, NA' \to S')$ and $(l_3, r_3) : (A \to PB, PB' \to S')$:
\begin{align*}
(\langle l_3, r_3 \rangle \circ \langle l_2, r_2 \rangle) \circ \langle l_1, r_1 \rangle 
&= \langle (N l_3)l_2, r_2(Nr_3) \rangle \circ \langle l_1, r_1 \rangle \\
&= \langle M((N l_3)l_2)l_1, r_1M(r_2(Nr_3)) \rangle \\
&= \langle (M N l_3)(M l_2)l_1, r_1(M r_2)(MNr_3)) \rangle \\
&= \langle l_3, r_3 \rangle \circ (\langle (M l_2)l_1, r_1(Mr_2) \rangle) \\
&= \langle l_3, r_3 \rangle \circ (\langle l_2, r_2 \rangle \circ \langle l_1, r_1 \rangle)
\end{align*}

\todo{To justify this kind of equational reasoning: We use that products in $\Set$ preserve colimits, so the coends can be floated to the outside and we may pick representatives for all 3 preoptics simultaneously. In the middle we include into the coend then immediately pick a representative, so by the universal property of coends we may as well use the representative we already have. It seems like this would also work in an enriched setting (if you really wanted it to)}

Finally, we have the unit laws. For $(l, r) : (S \to MA, MA' \to S')$ we calculate:
\begin{align*}
\id_{A, A'} \circ \langle l, r\rangle 
&= \langle \epsilon_A, \epsilon_{A'}^{-1} \rangle \circ \langle l, r\rangle \\
&= \langle (M\epsilon_A) l, r (M\epsilon_{A'}^{-1})\rangle \\
&= \langle (\lambda_M A) l, r (\lambda_M^{-1} A')\rangle \\
&= \langle l, r (\lambda_M^{-1} A') (\lambda_M A')\rangle \\
&= \langle l, r \rangle  \\
\langle l, r \rangle \circ \id_{S, S'} 
&= \langle l, r \rangle \circ \langle \epsilon_S, \epsilon_{S'}^{-1} \rangle  \\
&= \langle (Il)\epsilon_S, \epsilon_{S'}^{-1} (Ir) \rangle \\
&= \langle (\lambda_M S)l, r (\lambda_{S'}^{-1} S') \rangle \\
&= \langle l, r (\lambda_{S'}^{-1} S')(\lambda_M S') \rangle \\
&= \langle l, r \rangle
\end{align*}
Where in both cases we have used the coend relations to move $\lambda_M$ to the other side.
\end{proof}

In \cite[Section 6]{Doubles} a very similar category was considered. The \emph{double} $\Double_l \C$ of a monoidal category $(\C, \otimes, I)$ is \todo{the opposite category of} a special case of the above, where $\C$ is allowed to act on itself via left tensor. We will denote this case by $\PreOptic_\otimes$.

\begin{proposition}
For any $\C$ and $\M$, there is a functor $(-, -)_* : \C \times \C^\op \to \PreOptic_\M$, which is given on objects by $(S, S')_* = (S, S')$ and for a morphism $(f, g) : (S, S') \to (A, A')$ by $(f, g)_* = \langle \epsilon_A f, g \epsilon_{A'}^{-1} \rangle$.
\end{proposition}
\begin{proof}
This preserves identities, as the identity on an object $(S, S')$ in $\PreOptic_\M$ is defined to be exactly $\langle \epsilon_S, \epsilon_{S'}^{-1} \rangle$.

To check functoriality, suppose we have $(f, g) : (S, S') \to (A, A')$ and $(f', g') : (A, A') \to (B, B')$ in $\C \times \C^\op$. Then:
\begin{align*}
(f', g')_* \circ (f, g)_* 
&= \langle \epsilon_B f', g' \epsilon_{B'}^{-1} \rangle \circ \langle \epsilon_A f, g \epsilon_{A'}^{-1} \rangle \\
&= \langle (I(\epsilon_B f'))\epsilon_A f, g \epsilon_{A'}^{-1} (I(g' \epsilon_{B'}^{-1}))\rangle && \text{(By definition of $\circ$)}\\
&= \langle (I\epsilon_B) (I f')\epsilon_A f, g \epsilon_{A'}^{-1} (I g')(I\epsilon_{B'}^{-1})\rangle && \text{(Functoriality of action of $I$)}\\
&= \langle (I\epsilon_B) \epsilon_B f' f, g g' \epsilon_{B'}^{-1} (I\epsilon_{B'}^{-1})\rangle && \text{(Naturality of $\epsilon$)}\\
&= \langle (\lambda_I B) \epsilon_B f' f, g g' \epsilon_{B'}^{-1} (\lambda_I^{-1}B')\rangle && \text{(Unitality of action)} \\
&= \langle \epsilon_B f' f, g g' \epsilon_{B'}^{-1} (\lambda_I^{-1}B') (\lambda_I B') \rangle && \text{(Coend relation)}  \\
&= \langle \epsilon_B f'f, g g' \epsilon_{B'}^{-1} \rangle \\
&= (f'f, gg')_*
\end{align*}
\end{proof}
\todo{Is this functor monoidal?}

\todo{Is this functor faithful? What if $I$ is terminal? No! Take $S = \emptyset$}

\todo{This doesn't give us the structure of a `framing', we can only lift vertical arrows to horizontal arrows facing one direction.}

\begin{theorem}
If $(\C, \otimes, I)$ is a symmetric monoidal category, then $\PreOptic_\otimes$ is also symmetric monoidal. \todo{braided?}
\end{theorem}
\begin{proof}
On objects, we define $(S, S') \otimes (T, T') := (S \otimes T, S' \otimes T')$. The unit is given by the object $(I, I)$. The action of $\otimes$ on morphisms is induced by
\begin{align*}
&\C(S, M \otimes A) \times \C(M \otimes A', S') \times \C(T, N \otimes B) \times \C(N \otimes B', T')\\
\to \quad&\C(S \otimes T, M \otimes A \otimes N \otimes B) \times \C(M \otimes A' \otimes N \otimes B', S' \otimes T') && \text{(action of $\otimes$ on morphisms in $\C$)}\\
\to \quad&\C(S \otimes T, M \otimes N \otimes A \otimes B) \times \C(M \otimes N \otimes A' \otimes B', S' \otimes T') && \text{($\otimes$ in $\C$ is symmetric)}\\
\to \quad&\int^{M \in \C} \C(S \otimes T, M \otimes A \otimes B) \times \C(M \otimes A' \otimes B', S' \otimes T') && \text{(inclusion into coend)}
\end{align*}
yielding a function $\PreOptic_\otimes((S, S'), (A, A')) \times \PreOptic_\otimes((T, T'), (B, B')) \to \PreOptic_\otimes((S \otimes T, S' \otimes T'), (A \otimes B, A' \otimes B'))$. 

Equationally, suppose we have representatives $(l, r) : (S \to M \otimes A, M \otimes A' \to S')$ and $(l', r') : (T \to N \otimes B, N \otimes B' \to T')$. Then:
\begin{align*}
\langle l, r \rangle \otimes \langle l', r' \rangle &:= \langle (M \otimes s_{A,N} \otimes B)(l \otimes l'), (r \otimes r')(M \otimes s_{A',N} \otimes B') \rangle
\end{align*}
For future convenience, note that 
\begin{align*}
\langle l, r \rangle \otimes (T, T') &= \langle (M \otimes s_{A,I} \otimes T)(l \otimes \lambda_T), (r \otimes \lambda^{-1}_{T'})(M \otimes s_{A',I}\otimes T') \rangle \\
&= \langle (\lambda^{-1}_M \otimes A \otimes T)(M \otimes s_{A,I} \otimes T)(l \otimes \lambda_T), (r \otimes \lambda^{-1}_{T'})(M \otimes s_{A',I}\otimes T')(\lambda_M \otimes A' \otimes T') \rangle \\
&= \langle l \otimes T, r \otimes T' \rangle
\end{align*}
Now we check functoriality. Suppose we have:
\begin{align*}
\langle l_1, r_1 \rangle : (S_1, S_1') &\hto (S_2, S_2') \\
\langle l_2, r_2 \rangle : (S_2, S_2') &\hto (S_3, S_3') \\
\langle p_1, q_1 \rangle : (T_1, T_1') &\hto (T_2, T_2') \\
\langle p_2, q_2 \rangle : (T_2, T_2') &\hto (T_3, T_3')
\end{align*}
with complements $M_1, M_2, N_1, N_2$ respectively. Then: \todo {(This is a mess. String diagram?)}
\begin{align*}
&(\langle l_2, r_2 \rangle \otimes \langle p_2, q_2 \rangle) \circ (\langle l_1, r_1 \rangle \otimes \langle p_1, q_1 \rangle) \\
= \; &\langle (M_2 \otimes s_{S_3,N_2} \otimes T_3)(l_2 \otimes p_2), (r_2 \otimes q_2)(M_2 \otimes s_{S_3',N_2} \otimes T_3') \rangle \\
&\circ \langle (M_1 \otimes s_{S_2,N_1} \otimes T_2)(l_1 \otimes p_1), (r_1 \otimes q_1)(M_1 \otimes s_{S_2',N_1} \otimes T_2') \rangle \\
= \; & \langle (M_1 \otimes N_1 \otimes M_2 \otimes s_{S_3,N_2} \otimes T_3)(M_1 \otimes N_1 \otimes l_2 \otimes p_2)(M_1 \otimes s_{S_2,N_1} \otimes T_2)(l_1 \otimes p_1), \\
&\;(r_1 \otimes q_1)(M_1 \otimes s_{S_2',N_1} \otimes T_2')(M_1 \otimes N_1 \otimes r_2 \otimes q_2)(M_1 \otimes N_1 \otimes M_2 \otimes s_{S_3',N_2} \otimes T_3') \rangle \\
= \; & \langle (M_1 \otimes N_1 \otimes M_2 \otimes s_{S_3,N_2} \otimes T_3)(M_1 \otimes s_{(M_2 \otimes S_3),N_1} \otimes T_3)(M_1 \otimes  l_2 \otimes N_1 \otimes p_2)(l_1 \otimes p_1), \\
&\;(r_1 \otimes q_1)(M_1 \otimes r_2 \otimes N_1 \otimes q_2)(M_1 \otimes s_{(M_2 \otimes S_3'),N_1} \otimes T_3')(M_1 \otimes N_1 \otimes M_2 \otimes s_{S_3',N_2} \otimes T_3') \rangle \\
= \; & \langle (M_1 \otimes N_1 \otimes M_2 \otimes s_{S_3,N_2} \otimes T_3)(M_1 \otimes s_{(M_2 \otimes S_3),N_1} \otimes T_3)((M_1 \otimes l_2)l_1\otimes (N_1 \otimes p_2)p_1), \\
&\;(r_1(M_1 \otimes r_2) \otimes q_1(N_1 \otimes q_2))(M_1 \otimes s_{(M_2 \otimes S_3'),N_1} \otimes T_3')(M_1 \otimes N_1 \otimes M_2 \otimes s_{S_3',N_2} \otimes T_3') \rangle \\
= \; & \text{\todo{(Add a $s_{M_2, N_1}$ to either side of the coend, then appeal to coherence)}} \\
= \; &\langle (M_1 \otimes M_2 \otimes s_{(N_1 \otimes N_2),S_3} \otimes T_3)((M_1 \otimes l_2)l_1\otimes (N_1 \otimes p_2)p_1) ,  \\
&\;(r_1(M_1 \otimes r_2) \otimes q_1(N_1 \otimes q_2))(M_1 \otimes M_2 \otimes s_{(N_1 \otimes N_2),S'_3} \otimes T'_3) \rangle \\
= \; &\langle (M_1 \otimes l_2)l_1, r_1(M_1 \otimes r_2) \rangle \otimes \langle (N_1 \otimes p_2)p_1, q_1(N_1 \otimes q_2) \rangle \\
= \; &(\langle l_2, r_2 \rangle  \circ \langle l_1, r_1 \rangle) \otimes (\langle p_2, q_2 \rangle \circ \langle p_1, q_1 \rangle)
\end{align*}

The structure morphisms are all lifted from the structure morphisms in $\C$:
\begin{align*}
\alpha_{(R, R'), (S, S'), (T, T')} &:= (\alpha_{R,S,T}, \alpha_{R',S',T'}^{-1})_* \\
\lambda_{(S, S')} &:= (\lambda_{(S, S')}, \lambda_{(S, S')}^{-1})_* \\
\rho_{(S, S')} &:= (\rho_{(S, S')}, \rho_{(S, S')}^{-1})_* \\
s_{(S, S'), (T, T')} &:= (s_{S, T}, s_{T', S'})_*
\end{align*}
 
Note that because $(S, S')_* = (S, S')$, required equations for $(-, -)_*$ to be a monoidal functor hold by definition (although we don't yet know that $\PreOptic_\otimes$ is monoidal). The pentagon and triangle axioms then hold in $\PreOptic_\otimes$, as they are the image of the same axioms in $\C \times \C^\op$ under $(-, -)_*$. The only remaining piece to verify is that these structure maps are natural in $\PreOptic_\otimes$. 

Let us check that $\alpha_{(R, R'), (S, S'), (T, T')}$ is natural in $(R, R')$. Suppose $f : (Q, Q') \hto (R, R')$ has representative $(l, r) : (Q \to M \otimes R, M \otimes R' \to Q')$, then:
\begin{align*}
&\alpha_{(R, R'), (S, S'), (T, T')} \circ (f \otimes (S, S')) \otimes (T, T') \\
&= (\alpha_{R,S,T}, \alpha_{R',S',T'}^{-1})_*  \circ (\langle l, r \rangle \otimes (S, S')) \otimes (T, T') \\
&= \langle\lambda_{R \otimes (S \otimes T)}\alpha_{R,S,T}, \alpha_{R',S',T'}^{-1} \lambda^{-1}_{(R' \otimes S') \otimes T'} \rangle  \circ \left\langle (l \otimes S) \otimes T, (r \otimes S') \otimes T' \right\rangle \\
&=\left\langle (M \otimes \lambda_{R \otimes (S \otimes T)}\alpha_{R,S,T})((l \otimes S) \otimes T), ((r \otimes S') \otimes T') (M \otimes \alpha_{R',S',T'}^{-1} \lambda^{-1}_{(R' \otimes S') \otimes T'} ) \right\rangle \\
&=\left\langle (M \otimes \alpha_{R,S,T})((l \otimes S) \otimes T), ((r \otimes S') \otimes T') (M \otimes \alpha_{R',S',T'}^{-1} )\right\rangle \\
&=\left\langle (l \otimes (S \otimes T))\alpha_{Q,S,T}, \alpha_{Q',S',T'}^{-1}(r \otimes (S' \otimes T')) \right\rangle \\
&= \langle (I \otimes l \otimes (S \otimes T)) \lambda_{Q \otimes (S \otimes T)}\alpha_{Q,S,T}, (I \times r \otimes (S' \otimes T')) \alpha_{Q',S',T'}^{-1} \lambda^{-1}_{Q' \otimes (S' \otimes T')}\rangle \\
&= \langle l \otimes (S \otimes T), r \otimes (S' \otimes T')\rangle \circ \left\langle \lambda_{Q \otimes (S \otimes T)}\alpha_{Q,S,T}, \alpha_{Q',S',T'}^{-1} \lambda^{-1}_{Q' \otimes (S' \otimes T')} \right\rangle \\
&= \langle l, r \rangle \otimes ((S, S') \otimes (T, T')) \circ (\alpha_{Q,S,T}, \alpha_{Q',S',T'}^{-1})_* \\
&= f \otimes ((S, S') \otimes (T, T')) \circ \alpha_{(Q, Q'), (S, S'), (T, T')} 
\end{align*}
\todo{There is some cheating going on here, composition actually adds an extra $\alpha$, but that wouldn't get in the way.}
\end{proof}

In the Haskell \texttt{lens} library, the action of the monoidal product on optics is denoted ``\texttt{alongside}'' for the product and ``\texttt{without}'' for the coproduct.

There is some further useful structure in $\PreOptic_\otimes$. The set of costates $(S, S') \hto (I, I)$ is isomorphic to $\C(S, S')$:
\begin{align*}
\PreOptic_\otimes((S, S'), (I, I)) 
&= \int^{M \in \C} \C(S, M \otimes I) \times \C(M \otimes I, S') \\
&\cong \int^{M \in \C} \C(S, M) \times \C(M, S') \\
&\cong \C(S, S')
\end{align*}
The identity morphism $\C(S, S)$ gives us a preoptic $\varepsilon : (S, S) \to (I, I)$ that we call the \emph{counit}. 

States are not as easy to describe in general.

\begin{proposition}
Suppose the monoidal unit $I$ of $\C$ is terminal. Then the set of states $(I, I) \hto (A, A')$ is isomorphic to $\C(I, A)$.
\end{proposition}
\begin{proof}
\begin{align*}
\PreOptic_\otimes((I,I), (A,A')) 
&= \int^{M \in \C} \C(I, M \otimes A) \times \C(M \otimes A', I) \\
&\cong \int^{M \in \C} \C(I, M \otimes A)
\end{align*}
Because the interior of this coend is mute in the contravariant position, the coend is equal to the colimit of the functor $\C(I, - \otimes A) : \C \to \Set$. But $\C$ has a terminal object, so $\int^{M \in \C} \C(I, M \otimes A) \cong \C(I, I \otimes A) \cong \C(I, A)$.
\end{proof}

\subsection{Optics}
In this section we will focus only on preoptics of the form $p : (S,S) \hto (A, A)$. In what follows, we abbreviate $\PreOptic_\M((S, S), (A, A))$ as $\PreOptic_\M(S, A)$ and $p : (S, S) \hto (A, A)$ as $p : S \hto A$.

For a fixed $M \in \M$, there is a map $\C(S, M A) \times \C(M A, S) \to \C(S, S)$ given by composition. This induces a map $outside : \PreOptic_\M(S, A) \to \C(S, S)$. By composition in the other direction we have a map $\C(S, M A) \times \C(M A, S) \to \C(M A, M A)$. Including this object $\C(M A, M A)$ into the coend $\int^{M \in \M} \C(M A, M A)$, we induce a map $inside : \PreOptic_\M(S, A) \to \int^{M \in \M} \C(M A, M A)$. 
\begin{definition}
An preoptic $p : S \hto A$ is an \emph{optic} if $outside(p) = \id_S$, and $inside(p) = \langle t A \rangle$ for some $t : M \to M$ in $\M$.
\end{definition}

\begin{proposition}
\label{prop-onthenose}
If $p : S \to A$ is an optic, then there exists a representative $p = \langle l, r \rangle$ such that $rl = \phi A$ on the nose for some $\phi : M \to M$.
\end{proposition}
\begin{proof}
\todo{This is painful but seems important. There may be a more general principle at work here but I couldn't see it.}

\todo{Put in an appendix?}

There is a well known formula for the coend as the coequaliser in the diagram
\[
\begin{tikzcd}
\coprod_{M \to N} P(N, M) \ar[r,shift left=.75ex]  \ar[r,shift right=.75ex] & \coprod_{M \in \M} P(M, M) \ar[r] & \int^{M \in \M} P(M, M)
\end{tikzcd}
\]
where $S$ is some functor $\M^\op \times \M \to \E$.

In our case $\E = \Set$, and unwinding the coequaliser this states that $\int^{M \in \M} \C(M A, M A)$ is a quotient of the set $\coprod_{M \in \M} \C(M A, M A)$. Two morphisms $f : M A \to M A$ and $g : N A \to N A$ are identified when there exists a $k : N A \to M A$ and natural transformation $\phi : M \to N$ such that the diagram
\[
\begin{tikzcd}
M A \ar[r, "\phi_A"] \ar[d, "f", swap] & N A \ar[d, "g"] \ar[dl, "k"] \\
M A \ar[r, "\phi_A", swap] & N A
\end{tikzcd}
\]
commutes. This gives a binary relation $f \rightsquigarrow g$ that is not likely to be symmetric or transitive. The coend is obtained by identifying all morphisms related in this way, so in all, $\langle f \rangle = \langle g \rangle$ iff there exists a zigzag $f \leftrightsquigarrow u_1 \leftrightsquigarrow \dots \leftrightsquigarrow u_n \leftrightsquigarrow g$ where the relation may apply in either direction. \todo{replace this with $f = u_1 \leftrightsquigarrow \dots u_n = g$ for clarity}

If $f \rightsquigarrow g$ then $f^2 \rightsquigarrow g^2$: this is clear by stacking the above diagram, and then taking $k' = fk = kg$. This argument extends to show that $f^n \rightsquigarrow g^n$ for any $n$.

Similarly unwinding the coequaliser for $\int^{M \in \M} \C(S, M A) \times \C(M A, S)$, two pairs $(l, r) : (S \to M A, MA \to S)$ and $(l', r') : (S \to NA, NA \to S)$ in $\coprod_{M \in \M} \C(S, M A) \times \C(M A, S)$ are identified when there exists a $\phi : M \to N$ such that
\[
\begin{tikzcd}
M A \ar[r, "\phi_A"] \ar[d, "r", swap] & N A \ar[d, "r'"] \\
S \ar[r, equal] \ar[d, "l", swap] & S \ar[d, "l'"]  \\
MA \ar[r, "\phi_A"] & N A
\end{tikzcd}
\]
commutes. \todo{put the $S$ on the outside?}

We will now prove the result by induction. Because $inside(p) = \langle t \rangle$, there is a representative $\langle l, r \rangle$ so that $\langle lr \rangle = \langle t \rangle$. There therefore must exist a finite chain of relations $lr \leftrightsquigarrow u_1 \leftrightsquigarrow \dots \leftrightsquigarrow u_n \leftrightsquigarrow t$.
Supposing that the first relation faces rightwards, we have a diagram:
\[
\begin{tikzcd}
M A \ar[r, "\phi_A"] \ar[d, "r", swap] & N A \ar[dd, "u_1"] \ar[ddl, "k"] \\
S \ar[d,"l",swap] & \\
M A \ar[r, "\phi_A", swap] & N A
\end{tikzcd}
\]
Note that $r = rlr = rk\phi_A$. We therefore have the commutative diagram
\[
\begin{tikzcd}
M A \ar[r, "\phi_A"] \ar[d, "r", swap] & N A \ar[d, "rk"] \\
S \ar[r, equal] \ar[d, "l", swap] & S \ar[d, "\phi_Al"] \\
MA \ar[r, "\phi_A", swap] & N A
\end{tikzcd}
\]
where the composite of the morphisms on the right is $\phi_A l r k = \phi_A k \phi_A k = u_1^2$. Composing in the other direction we have $r k \phi_A l = rlrl = \id_S$. We have shown that whenever $rl = \id_S$ and $lr \rightsquigarrow u_1$, then there exist $l_1$ and $r_1$ so that $r_1l_1 = \id_S$, $l_1r_1 = u_1^2$ and $(l, r) \rightsquigarrow (l_1, r_1)$. A symmetric argument shows that if instead $lr \leftsquigarrow g$, there exist $l_1$ and $r_1$ so that $r_1l_1 = \id_S$, $l_1r_1 = u_1^2$ and $(l, r) \leftsquigarrow (l_1, r_1)$.

We can now inductively apply the above argument to the chain $u_1^2 \leftrightsquigarrow \dots \leftrightsquigarrow u_n^2 \leftrightsquigarrow t^2 = t$, obtained by squaring each morphism in the original chain. We eventually obtain a chain $(l, r) \leftrightsquigarrow (l_1, r_1)\leftrightsquigarrow \dots \leftrightsquigarrow (l_n, r_n) \leftrightsquigarrow (l_t, r_t)$, where $r_tl_t = \id_S$ and $l_t r_t = t$. This pair $(l_t, r_t)$ is the required representative.
\end{proof}

The above argument has a similar form to some that appear in \cite{OnTheTrace}, which considered (among other things) coends of the form $\int^{c \in \C} \C(c, Fc)$ for an endofunctor $F : \C \to \C$.

In view of this Proposition, whenever we posit a representative $\langle l, r \rangle$ of an optic, we may assume that $lr = \phi A$ on the nose, for some $\phi : M \to M$.

\begin{proposition}
There is an subcategory $\Optic_\M$ of $\PreOptic_\M$ given by objects of the form $(X, X)$ and optics between them.
\end{proposition}
\begin{proof}
The identity preoptic $(\id_X, \id_X)_*$ is clearly an optic. 

Suppose we have two optics $\langle l, r \rangle : R \hto S$ and $\langle l', r' \rangle : S \hto A$, where $l : R \to MS$ and $l' : S \to NA$, and such that $lr = \phi S$ and $l'r' = \psi A$. Then the composite $\langle (Ml')l, r (Mr')  \rangle$ is also an optic:
\begin{align*}
r (Mr')(Ml')l &= r M(r'l') l = rl = \id_S
\intertext{and}
(Ml')l r (Mr') &= (Ml')(\phi S)(Mr') = (\phi NA) M(l')M(r') = (\phi NA) M(l'r') = (\phi NA) M(\psi A) = \phi(\psi A)
\end{align*}
by the naturality of the action of $\M$.
\end{proof}

\subsection{Examples}

\subsubsection{Lenses}

Lenses form the ur-example of a category of optics.

\begin{definition}
Suppose $\C$ has finite products, and let $\C$ act on itself via the bifunctor $\times : \C \times \C \to \C$. The \emph{category of lenses} $\Lens$ in $\C$ is the category of optics with respect to this action: $\Lens = \Optic_\times$.
\end{definition}

As remarked in the introduction \todo{(todo: remark in the introduction)}, the preoptics for this action are exactly pairs of $\fget$ and $\fput$ functions.
\begin{align*}
\PreOptic_\times(S, A) &= \int^{M \in \M} \C(S, M \times A) \times \C(M \times A, S) \\
&\cong \int^{M \in \M} \C(S, M) \times \C(S, A) \times \C(M \times A, S) && \text{(universal property of product)} \\
&\cong \C(S, A) \times \C(S \times A, S) && \text{(co-Yoneda)}
\end{align*}

Explicitly this isomorphism states that, given a prelens $\langle l, r \rangle : S \hto A$, the corresponding $\fget$ and $\fput$ functions are given by $\fget = \pi_2 l$ and $\fput = r (\pi_1 l \times A)$. In the other direction, given $\fget$ and $\fput$, the corresponding element in the coend has representative $\langle l, r \rangle$, where $l = [\id_s, \fget]$ and $r = \fput$. Note that even if the $\fget$ and $\fput$ correspond to a law-abiding lens, this particular representative will not be an isomorphism.

\begin{proposition}
\label{prop-OpticImpliesLensLaws}
Suppose $\langle l, r \rangle : S \hto A$ is a pre-lens such that $rl = \id_S$ and $\langle lr \rangle = \langle \phi \times A \rangle$ as an element of $\int^{M \in \M} \C(M \times A, M \times A)$, for some $\phi : M \to M$. Then $p$ obeys the three lens laws.
\end{proposition}
\begin{proof}
By Proposition \ref{prop-onthenose}, we may assume that $lr = \phi \times A$ on the nose.

Let $\fget = \pi_2 l$ and $\fput = r (\pi_1 l \times A)$. Verification of the laws is straightforward:
\begin{align*}
\fget \; \fput 
&= \pi_2 l r (\pi_1 l \times A) \\
&= \pi_2 (t \times A) (\pi_1 l \times A) \\
&= \pi_2 (t \pi_1 l \times A) \\
&= \pi_2 \\
\fput [\id_S, \fget] 
&= r (\pi_1 l \times A) [\id_S, \pi_2 l] \\
&= r [\pi_1 l, \pi_2 l] \\
&= rl \\
&= \id_S \\
\fput (\fput \times A) 
&= r (\pi_1 l \times A) (r (\pi_1 l \times A) \times A) \\
&= r (\pi_1 l  r (\pi_1 l \times A) \times A) \\
&= r (\pi_1 (t\pi_1 l \times A) \times A) \\
&= r (t\pi_1 l \times A) \pi_{1,3} \\
&= r lr(\pi_1 l \times A) \pi_{1,3} \\
&= r(\pi_1 l \times A) \pi_{1,3} \\
&= \fput \, \pi_{1, 3}
\end{align*}

\todo{Written out at \cite{IsomorphismLensesPost}, definitely not the first place though.}
\end{proof}

\begin{proposition}
Suppose there exists a map $x : 1 \to A$. Then if $\fget : S \to A$ and $\fput : S \times A \to S$ satisfy the three lens laws then the corresponding $p : S \hto A$ is a lens.
\end{proposition}
\begin{proof}
Recall that the corresponding optic $p : S \hto A$ has as a representative $\langle [\id_s, \fget], \fput \rangle$.

Let $\fput_x : S \to S$ be the composite $S \to S \times 1 \xrightarrow{x} S \times A \xrightarrow{\fput} S$. The commutative diagram
\[
\begin{tikzcd}
S \times A \ar[r, "\fput_x \times A "] \ar[d, "lr", swap] & S \times A \ar[d, "\fput_x \times A"] \ar[dl, "lr"] \\
S \times A \ar[r, "\fput_x \times A", swap] & S \times A
\end{tikzcd}
\]
is a witness that $\langle lr \rangle = \langle \fput_x \times A \rangle$ in $\int^{M \in \M} \C(M A, M A)$.
\todo{Is there some property of $\times$ used here we can abstract? Having a map $1 \to A$ guarantees that $- \times A$ is faithful, is this the trick?}
\end{proof}

In favourable conditions, we can upgrade this result to show that a lens $S \hto A$ in fact implies that $S \cong M \times A$ for some $M$. This is similar to the result proved in \cite[Corollary 13]{AlgebrasAndUpdateStrategies} for $\Set$.

\begin{proposition}
Suppose $\C$ has finite limits and that there is a morphism $x : 1 \to A$. Then if $\fget : S \to A$ and $\fput : S \times A \to S$ satisfy the three lens laws then there exists $C \in \C$ with $S \cong C \times A$.
\end{proposition}
\begin{proof}
Set $C$ to be the pullback of $\fget$ along $x$, so there is a map $i : C \to S$ with $\fget \, i = x !_C$. There is also a map $j : S \to C$ induced by the following diagram:
\[
\begin{tikzcd}
S \ar[ddr, bend right = 20] \ar[dr, "j", dashed] \ar[r, "{[\id_S, x !_S]}"] & S \times A \ar[dr, "\fput"] & \\
& C \ar[r, "i"] \ar[d] \arrow[dr, phantom, "\lrcorner", very near start] & S \ar[d, "\fget"] \\
& 1\ar[r, "x", swap] & A
\end{tikzcd}
\]
which commutes by the $\fput\fget$ law. Note that $ji = \id_C$ by the universal property of pullbacks. The claim is that $\fput (i \times A) : C \times A \to S$ and $[j,\fget] : S \to C \times A$ are mutual inverses. This is easily checked:
\begin{align*}
\fput (i \times A)[j,\fget] &= \fput [ij,\fget] \\
&= \fput [\fput [\id_S, x!_S],\fget] && \text{(by definition of $j$)} \\
&= \fput [\id_S,\fget] && \text{(by $\fput\fput$)} \\
&= \id_S && \text{(by $\fget\fput$)}
\intertext{and}
[j,\fget]\fput (i \times A) &= [j\fput (i \times A),\fget\,\fput (i \times A)] && \text{(by universal property of product)} \\
&= [j\fput (i \times A), \pi_2 (i \times A)] && \text{(by $\fput\fget$)} \\
&= [j\fput (i \times A), \pi_2] && \\
&= [jij\fput (i \times A), \pi_2] && \\
&= [j\fput [\id_S,x !_S] \fput (i \times A), \pi_2] && \\
&= [j\fput [\id_S, x !_S] \pi_1 (i \times A), \pi_2] && \text{(by $\fput\fput$ \todo{todo: detail})}\\
&= [jij \pi_1 (i \times A), \pi_2] && \\
&= [jiji \pi_1, \pi_2] && \\
&= [\pi_1, \pi_2] && \\
&= \id_{C \times A}
\end{align*}

Additionally, the commutative diagram
\[
\begin{tikzcd}
C \times A \ar[r, "i \times A"] \ar[d, "\fput (i \times A)", swap] & S \times A \ar[d, "\fput"] \\
S \ar[r, equal] \ar[d, "{[j,\fget]}", swap] & S \ar[d, "{[\id_S, \fget]}"]  \\
C \times A \ar[r, "i \times A", swap] & S \times A
\end{tikzcd}
\]
is a witness that $\langle \fput (i \times A), [j,\fget] \rangle = \langle \fput, [\id_S, \fget] \rangle$ as elements of $\PreOptic_\M(S, A)$.
\end{proof}

\subsubsection{Prisms}
Prisms are dual to lenses:

\begin{definition}
Suppose $\C$ has finite coproducts, and let $\C$ act on itself via the bifunctor $\sqcup : \C \times \C \to \C$. The \emph{category of prisms} in $\C$ is the category of optics with respect to this action: $\Prism = \Optic_\sqcup$.
\end{definition}

Just as lenses are constructed by two concrete maps $\fget : S \to A$ and $\fput : S \times A \to S$, prisms are constructed by two maps $\freview : A \to S$ and $\fmatching : S \to S \sqcup A$. These names are taken from the Haskell \texttt{lens} library.
\begin{align*}
\PreOptic_\sqcup(S, A) &= \int^{M \in \M} \C(S, M \sqcup A) \times \C(M \sqcup A, S) \\
&\cong \int^{M \in \M} \C(S, M \sqcup A) \times \C(M, S) \times \C(A, S) && \text{(universal property of coproduct)} \\
&\cong \C(S, S \sqcup A) \times \C(A, S) && \text{(co-Yoneda)}
\end{align*}
So, if we are given a pre-prism $\langle l, r \rangle : (S, S) \hto (A, A)$ then associated $\freview$ and $\fmatching$ morphisms are given by $\freview = r \inr$ and $\fmatching = (r\inl \sqcup \id_A)l$

The laws for prisms are the obvious duals to the lens laws. 
\begin{align*}
\fmatching \; \freview &= \inr \\
[\id_S, \freview] \fmatching &= \id_S \\
(\fmatching \sqcup \id_A) \fmatching &= \mathrm{in}_{1,3} \, \fmatching
\end{align*}

In the \texttt{lens} library documentation the third law is missing, due to the following:

\begin{proposition}
When $\C = \Set$\todo{, and probably more generally,} the third law is implied by the other two.
\end{proposition}
\begin{proof}
The key is that in $\Set$, for any map $f : X \to Y$, the set $Y$ splits into the coproduct of $\im f$ and its complement. The first law implies that $\freview$ is injective, so $S \cong C \sqcup A$ for some complement $C$. Identifying $A$ with its image in $S$, the second law implies that if $a\in A \subset S$ then $\fmatching(a) = \inr(a)$ and if $c\in C \subset S$ then $\fmatching(c) = \inl(c)$. The third law then follows easily by checking both cases.
\end{proof}

\begin{proposition}
\label{prop-OpticImpliesPrismLaws}
If $\langle l, r \rangle : S \hto A$ is a prism then the associated $\fmatching$ and $\freview$ functions satisfy the above laws.
\end{proposition}
\begin{proof}
\todo{Again it seems it is good enough if we only have $lr = t \sqcup A$}
\begin{align*}
\fmatching \; \freview 
&= (r\inl \sqcup \id_A)l r \inr \\
&= (r\inl \sqcup \id_A) \inr \\
&= \inr \\
[\id_S, \freview] \fmatching
&= [\id_S, r \inr] (r\inl \sqcup \id_A)l \\ 
&= [r\inl, r \inr] l \\ 
&= rl \\
&= \id_S
\end{align*}
\end{proof}

\todo{unfinished}

\subsubsection{Traversals}
\subsubsection{Setters}

To define setters we must first recall the notion of a strong functor.

\todo{Taken from http://cs.ioc.ee/~tarmo/ssgep15/ssgep-1a.pdf}
\begin{definition}
\todo{Again I am not fussing with associators/unitors, this is probably bad. Also this definition should work fine of we instead use an action of some $\M$ on $\C$.} Suppose $(\C, \otimes, I)$ is a monoidal category. A \emph{(left) strength} for an endofunctor $F : \C \to \C$ is a natural transformation
\begin{align*}
\theta_{A,B} : A \otimes F B \to F(A \otimes B)
\end{align*}
such that both
\[
\begin{tikzcd}
A \otimes B \otimes F C \ar[r, "A \otimes \theta_{B,C}"] \ar[d, "\theta_{A \otimes B, C}" left]  & A \otimes F(B\otimes C) \ar[d, "\theta_{A, B\otimes C}" right] \\
F(A \otimes B \otimes C) \ar[r, equals]& F(A \otimes B \otimes C)
\end{tikzcd}
\]
and
\[
\begin{tikzcd}
1 \otimes F A \ar[r, "s_{1,A}"] \ar[d, "\cong" left]  & F(1 \otimes A) \ar[d, "\cong" right] \\
F A \ar[r, equals] & F A
\end{tikzcd}
\]
commute. A \emph{strong functor} is a functor equipped with a strength. A \emph{strong natural transformation} $\tau : (F,\theta) \to (G,\theta')$ is a natural transformation that respects the strength:
\[
\begin{tikzcd}
A \otimes F B \ar[r, "\theta_{A,B}"] \ar[d, "A \otimes \tau_B" left]  & G(A \otimes B) \ar[d, "\tau_{A \otimes B}" right] \\
A \otimes G B \ar[r, "\theta'_{A, B}"]& G(A \otimes B)
\end{tikzcd}
\]

There is an evident category $\Strong(\C)$ of strong endofunctors and strong natural transformations, and a forgetful functor $U : \Strong(\C) \to [\C, \C]$.
\end{definition}

Every endofunctor on $(\Set, \times, 1)$ has a unique strength, so this concept is invisible for Haskell functors.

\begin{lemma}
Suppose $\C$ is cartesian closed and $F : \C \to \C$ is strong with respect to $\times$. Then for any objects $A, B \in \C$, \[\C(A, FB) \cong \Strong_\C(\times)(A \times \homC(B, -), F)\]
\end{lemma}
\begin{proof}
Given a $f \in \C(A, FB)$, let $\eta$ be the natural transformation with components
\begin{align*}
A \times \homC(B, C) \xrightarrow{f \times \id} FB \times \homC(B, C) \xrightarrow{\theta_{\homC(B, C), B}}  F(B \times \homC(B, C)) \xrightarrow{F\epsilon} FC
\end{align*}
where $\theta$ is the strength for $F$ and $\epsilon$ is the evaluation map.

Conversely, given an $\eta : A \times \homC(B, -) \to F$, the component at $B$ is a function $g : A \times \homC(B, B) \to FB$; we recover $f : A \to FB$ as $f = g (\id_A, \const_{\id_B})$.

\todo{These are inverses.}
\end{proof}

This is a generalisation of \cite[Proposition 2.2]{SecondOrderFunctionals}

\todo{unfinished}

\subsubsection{Affine Traversals}

\subsection{Equivalence of Optics}

One issue with the definition of optic given in the last section is that it could be difficult to determine when two optics $\langle l, r \rangle, \langle l', r' \rangle : S \hto A$ are equal. For some choices of $\M$, however, we can reduce a zigzag of relations to a single one.

\todo{This may work in general with some huge number of conditions: something like $-A$ reflects monomorphisms and preserves binary coproducts, all subobjects in $\M$ have complements.}

\begin{proposition}
Suppose all epimorphisms split in $\C$, and that $- \times A : \C \to \C$ reflects epis. Then two lenses $\langle l, r \rangle$ and $\langle l', r' \rangle :  S \hto A$ with complements $M$ and $N$ respectively are equal iff there exists an morphism $\phi : M \to N$ such that $(\phi \times A)l = l'$ and $r = r' (\phi \times A)$.
\end{proposition}
\begin{proof}
The backward direction is clear: the morphism $\phi$ is a witness that $\langle l, r \rangle \rightsquigarrow \langle l', r' \rangle$, so the two optics are equal.

%note that $p$ and $q$ must have identical $\fget$ and $\fput$ functions:
%\begin{align*}
%\fget &= \pi_2 l = \pi_2 l' \\
%\fput &= r (\pi_1 l \times A) = r' (\pi_1 l' \times A)
%\end{align*}
For the forward direction, note that the map $(\pi_1 l \times A)$ is a retraction: it has the section $(r, \pi_2)$:
\begin{align*}
(\pi_1 l \times A)(r, \pi_2) = (\pi_1 l r, \pi_2) = (\pi_1, \pi_2) = \id_{S \times A}
\end{align*}
Now by assumption, $\pi_1 l : S \to M$ is an epi, so has a retraction $f : M \to S$. The claim is that $\phi = \pi_1 l' f$ is the required morphism $M \to N$.

The situation is summarised in the following diagram, which other than the dashed line, we know commutes.
\[
\begin{tikzcd}
M \times A \ar[r, "{f \times A}", dashed, bend left = 50] \ar[d, "r", swap] & S \times A \ar[d, "\fput"] \ar[r, "{\pi_1 l' \times A}"] \ar[l, "{\pi_1 l \times A}", swap] & N \times A \ar[d, "r'"] \\
S \ar[r, equal] \ar[d, "l", swap] & S \ar[d, "{[\id_S, \fget]}"] \ar[r, equal] & S \ar[d, "l'"] \\
M \times A & S \times A \ar[r] \ar[l] & N \times A
\end{tikzcd}
\]
We can now calculate:
\begin{align*}
r = r(\pi_1 l f \times A) = \fput (f \times A) = r' (\pi_1 l' f \times A) = r'(\phi \times A)
\end{align*}
as required. The equation relating $l$ and $l'$ follows immediately as they are the inverses of $r$ and $r'$.
\end{proof}

The conditions of the above Proposition hold in $\Set$, so long as $A$ is not empty.

% Old attempt:
%\begin{definition}
%A functor $F : \C \to \D$ satisfies the \emph{splitting morphism condition} (SM) if, for every $f : A \to B$ in $\C$ such that $Ff$ is a split monomorphism, there exists a $g : B \to A$ such that $F(gf) = \id_{FA}$.
%
%Similarly, $F$ satisfies the \emph{splitting epimorphism condition} (SE) if for every $f : A \to B$ in $\C$ such that $Ff$ is a split epimorphism, there exists a $g : B \to A$ such that $F(fg) = \id_{FB}$.
%\end{definition}
%
%These definitions are identical to ones given in a (as far as I can tell totally unrelated) paper on triangulated categories \cite{ObjectiveTriangleFunctors}. The conditions are not too difficult to satisfy:
%
%\begin{proposition}[\todo{cite}]
%If $F$ is full then it satisfies (SM) and (SE).
%\end{proposition}
%\begin{proof}
%
%\end{proof}
%
%\begin{proposition}[\todo{cite}]
%If $F$ is faithful then it satisfies (SM) and (SE).
%\end{proposition}
%\begin{proof}
%
%\end{proof}
%
%\begin{theorem}
%Suppose we have two optics $\langle l, r \rangle$ and $\langle l', r' \rangle \in \Optic_\M(S, A)$ with complements $M$ and $N$ respectively, and suppose $\M$ has the property that the ``evaluation-at-$A$'' functor $(-A) : \M \to \C$ satisfies (SM) and (SE). Then the two optics are equal iff there exists an morphism $\phi : M \to N$ such that $\phi_A l = l'$ and $r = r' \phi_A$.
%\end{theorem}
%\begin{proof}
%The backward direction is clear: the morphism $\phi$ is a witness that $\langle l, r \rangle \rightsquigarrow \langle l', r' \rangle$, so the two optics are equal.
%
%For the forward direction, as in Proposition \ref{prop-onthenose} we induct on a zigzag \[(l, r) = (l_1, r_1) \leftrightsquigarrow \dots \leftrightsquigarrow (l_n, r_n) = (l', r'),\] where each pair $(l_i, r_i)$ has complement $M_i$. As a base case, we can take the identity $\id_M : M \to M = M_1$. Now suppose we already have a morphism $\phi : M \to M_{i-1}$ that satisfies $\phi_A l = l_{i-1}$ and $r = r_{i-1} \phi_A$. There are two cases to consider:
%
%If $(l_{i-1}, r_{i-1}) \rightsquigarrow (l_i, r_i)$, then by definition we have a $\psi : M_{i-1}\to M_i$ that satisfies the equations $\psi_A l_{i-1} = l_i$ and $r_{i-1} = r_i \psi_A$. Therefore the composite $\psi\phi$ satisfies $(\psi\phi)_A l = \psi_A l_{i-1} = l_i$ and $r = r_{i-1} \phi_A = r_i (\psi \phi)_A$, so $\psi\phi : M \to M_i$ is the required morphism.
%
%If instead $(l_{i-1}, r_{i-1}) \leftsquigarrow (l_i, r_i)$, we have a morphism in the opposite direction $\psi : M_i \to M_{i-1}$, with $\psi_A l_i = l_{i-1}$ and $r_i = r_{i-1} \psi_A$, as shown in the following commutative diagram
%\[
%\begin{tikzcd}
%M A \ar[r, "\phi_A"] \ar[d, "r", swap] & M_{i-1} A \ar[d, "r_{i-1}"] & M_i A \ar[l, "\psi_A", swap] \ar[d, "r_i"]\\
%S \ar[r, equal] \ar[d, "l", swap] & S \ar[r, equal] \ar[d, "l_{i-1}"] & S \ar[d, "l_i"] \\
%M A \ar[r, "\phi_A", swap] & M_{i-1} A & M_i A \ar[l, "\psi_A"]
%\end{tikzcd}
%\]
%To construct a morphism $M \to M_i$ we have to perform a slightly awkward switchback maneuver. First note that $\phi_A$ is a split monomorphism, as $(l r_{i-1}) \phi_A = lr = \id_{MA}$.
%
%By condition (SM), therefore, there is a morphism $\gamma : M_{i-1} \to M$ such that $\gamma_A \phi_A = \id_{MA}$. But now, the composite $\gamma_A \psi_A : M_i \to M$ is a split epimorphism:
%\begin{align*}
%(\gamma_A \psi_A)(l_i r) = \gamma_A l_{i-1} r = \gamma_A \phi_A l r = \gamma_A \phi_A = \id_{MA}
%\end{align*}
%and by condition (SE), there is a morphism $\theta : M \to M_i$ with $\gamma_A \psi_A \theta_A = \id_{MA}$. It remains to show that $\theta_A l = l_i$ and $r = r_i \theta_A$. First some intermediate steps:
%\begin{align*}
%r \gamma_A = r_{i-1} l_{i-1} r \gamma_A = r_{i-1} \phi_A l r \gamma_A = r_{i-1} \phi_A \gamma_A = 
%\end{align*}
%So finally:
%\begin{align*}
%r &= r \gamma_A \psi_A \theta_A	
%\end{align*}
%\todo{picture}
%\end{proof}

\subsection{Change of Action}

\section{Profunctor Optics}

\subsection{Profunctors}

\todo{recall definitions} We use $I$ to refer to the identity profunctor $\C(-,{=})$, and $\odot$ for horizontal profunctor composition. Profunctor composition is written in diagramatic order.

\subsection{Tambara Modules}
The following section generalises definitions that first appeared in \cite{Doubles}.
\begin{definition}
Suppose $\C$ is acted on by $(\M, \otimes, I)$ and let $P \in \Prof(\C, \C)$ be a profunctor. A \emph{Tambara module structure for $\M$} on $P$ is a natural family of maps:
\begin{align*}
\alpha_{A,B,M} : P(A,B) \to P(MA, MB)
\end{align*}
such that $\alpha_{A,B,I} = \id_{P(A,B)}$, and the following diagram commutes:
\[
\begin{tikzcd}
P(A,B) \ar[r, "\alpha_{A,B,M}"] \ar[dr, "\alpha_{A, B, N \otimes M}" below left] & P(MA, MB) \ar[d, "\alpha_{MA, MB, N}" right] \\
& P(NMA, NMB)
\end{tikzcd}
\]
\end{definition}

Note that the hom profunctor $I := \C(-, =) : \C^\op \times \C \to \Set$ has a canonical Tambara module structure for any $\M$, given by functoriality.

If $P, Q \in \Prof(\C, \C)$ are equipped with module structures $\alpha$ and $\beta$ respectively, there is a canonical module structure on $P \odot Q$. Given $M \in \M$ and $A,B \in \C$, the structure map $(\alpha \odot \beta)_{A,B,M}$ is induced by
\begin{align*}
&P(A,C) \times Q(C,B)  \\
\xrightarrow{\alpha_{A,C,M} \times \beta_{C,B,M}} \quad& P(MA, MC) \times Q(MC, MB) \\
\xrightarrow{\copr_{MC}} \quad&\int^{C \in \C} P(MA, C) \times Q(C, MB) \\
= \quad&(P \odot Q)(MA, MB)
\end{align*}

\begin{definition}
There is a category $\Tamb_\C(\M)$ of Tambara modules and natural transformations that respect the extra structure, in the sense that for any $l : P \to Q$, the diagram
\[
\begin{tikzcd}
P(A,B) \ar[r, "\alpha_{A,B,M}"] \ar[d, "l_{A,B}" left] & P(MA, MB) \ar[d, "l_{MA, MB}" right] \\
Q(A,B) \ar[r, "\beta_{A,B,M}"] & Q(MA, MB)
\end{tikzcd}
\]
commutes.
\end{definition}

This category is monoidal with respect to $\odot$ as given above, and the monoidal unit is given by $I$. There is an evident forgetful functor $U : \Tamb_\C(\M) \to \Prof(\C, \C)$ that is strong monoidal. This forgetful functor has both a left and right adjoint; particularly important for us is the left adjoint:

\begin{definition}
Let $\Pastro_\M : \Prof(\C, \C) \to \Tamb_\C(\M)$ be the functor \todo{change notation? drop $\C$ if we use the same one the whole time?}
\begin{align*}
\Pastro_\M(P) := \int^{M \in \M}  \C(-, M{=}) \odot P \odot \C(M-, {=}) 
\end{align*}
Or, in other words, 
\begin{align*}
\mathrm{Pastro}_\C(\M)(P)(A,B) := \int^{M \in \M} \int^{C,D \in \C} \C(A, MC) \times P(C,D) \times  \C(MD, B)
\end{align*}
The module structure $\alpha_{A,B,M} : \Pastro_\M(P)(A,B) \to \Pastro_\M(P)(MA, MB)
$ is given by map induced by 
\begin{align*}
&\int^{C,D \in \C} \C(A, NC) \times P(C,D) \times  \C(ND, B) \\
\xrightarrow{\text{functoriality}} \quad& \int^{C,D \in \C} \C(MA, MNC) \times P(C,D) \times  \C(MND, MB) \\
\xrightarrow{\copr_{MN}} \quad&\int^{N \in \M} \int^{C,D \in \C} \C(MA, NC) \times P(C,D) \times  \C(ND, MB) \\
= \quad&\Pastro_\M(P)(MA, MB)
\end{align*}
for all $N \in \M$.
\end{definition}

\begin{proposition}
$\Pastro_\M : \Prof(\C, \C) \to \Tamb_\C(\M)$ is left adjoint to $U : \Tamb_\C(\M) \to \Prof(\C, \C)$.
\end{proposition}
\begin{proof}
\todo{Could crunch through this manually, may be a neat way to do it using techniques from P/S}
\end{proof}

\begin{corollary}
$\Pastro_\M$ is oplax monoidal.
\end{corollary}
\begin{proof}
This follows abstractly as a left adjoint of a strong monoidal functor. \todo{reference?}
\end{proof}

\subsection{Preoptics}

\begin{definition}
For an object $A \in \C$, the \emph{exchange profunctor} $E_A$ is defined to be $\C(-, A) \times \C(A, {=})$.
\end{definition}

Given a profunctor, or indeed a Tambara module, we can evaluate it at any two objects of $\C$. This process is functorial in the choice of Tambara module, giving a functor $(U-)(A,A) : \Tamb_\C(\M) \to \Set$.

\begin{lemma}
\label{lemma-rep}
The functor $(U-)(A,A) : \Tamb_\C(\M) \to \Set$ is representable: there is a natural isomorphism
$(U-)(A,A) \cong \Tamb_\C(\M)(\Pastro_\M E_A, -)$
\end{lemma}
\begin{proof}
We have the chain of natural isomorphisms:
\begin{align*}
&(U-)(A,A) \\
\cong \;&\int_{X,Y \in \C} \Set(\C(X,A) \times \C(A,Y), (U-)(X,Y)) && \text{(by Yoneda (un)reduction twice)} \\
=\;&\int_{X,Y \in \C} \Set(E_A(X,Y), (U-)(X,Y)) && \text{(by definition)}\\
\cong \;&\Prof(E_A, U-) && \text{(natural transformations as ends)} \\
\cong \;&\Tamb_\C(\M)(\Pastro_\M E_A, -) && \text{(by adjointness)} 
\end{align*}
\end{proof}

We can now show that profunctor preoptics are precisely preoptics in the ordinary sense.

\begin{proposition}
\label{prop-profunctor-optics-are-optics}
\begin{align*}
[\Tamb_\C(\M), \Set]((U-)(A,A),(U-)(S,S)) &\cong \PreOptic_\M((S, S), (A, A))
\end{align*}
\end{proposition}
\begin{proof}
We have the chain of isomorphisms:
\begin{align*}
&[\Tamb_\C(\M), \Set]((U-)(A,A),(U-)(S,S)) \\
\cong \;&[\Tamb_\C(\M), \Set](\Tamb_\C(\M)(\Pastro_\M E_A, -), (U-)(S,S))  && \text{(by Lemma \ref{lemma-rep})}\\
\cong \;&(U\Pastro_\M E_A)(S,S)  && \text{(by Yoneda)} \\
= \;&\int^{M \in \M} \int^{C,D \in \C} \C(S, MC) \times E_A(C,D) \times \C(MD,S) \\
= \;&\int^{M \in \M} \int^{C,D \in \C} \C(S, MC) \times \C(C,A) \times \C(A,D) \times \C(MD,S) \\
\cong \;&\int^{M \in \M} \C(S, MA) \times \C(MA, S)  && \text{(by co-Yoneda twice)} \\
= \;&\PreOptic_\M((S, S), (A, A))
\end{align*}
\todo{This works for $\PreOptic_\M((S, S'), (A, A'))$ just fine, just define $E_{A,A'}$ in the obvious way.}

\todo{Flip the order of the isomorphisms top to bottom?}
\end{proof}

\begin{corollary}
A profunctor preoptic is determined by its component at $\Pastro_\M E_A$, and furthermore, this component is determined by its value on $\langle \id_A, \id_A, \id_A, \id_A \rangle \in (U \Pastro_\M E_A)(A, A)$.
\end{corollary}
\begin{proof}
This is the content of the first two isomorphisms above.

Explicitly, let $p_{\Pastro_\M E_A}(\langle \id_A, \id_A, \id_A, \id_A \rangle) = \langle l, \id_A, \id_A, r \rangle$, where $l : S \to M A$ and $r : M A \to S$. Then the component of $p$ at any other Tambara module $P$ is given by
\begin{align*}
p_P = (UP)(l,r) \alpha_{A,A,M}
\end{align*}
where $\alpha$ is the module structure for $P$.
\end{proof}



\subsection{Optics}

The next question is: can we characterise the profunctor preoptics that actually correspond to the lawful optics? The plan is to chase through the above isomorphisms to see where the lawful optics end up.

The exchange profunctor has a canonical comonoid structure. The comultiplication $\Delta : E_A \to E_A \odot E_A$ is given by the composite
\begin{align*}
&E_A(X,Y) \\
= \quad& \C(X, A) \times \C(A, Y) \\
\xrightarrow{[\pi_1, \const_{\id_A}, \const_{\id_A}, \pi_2]} \quad& \C(X, A) \times \C(A, A) \times \C(A, A) \times \C(A, Y) \\
\xrightarrow{\copr_A} \quad& \int^{C \in \C} \C(X, A) \times \C(A, C) \times \C(C, A) \times \C(A, Y) \\
= \quad& \int^{C \in \C} E_A(X,C) \times E_A(C,Y) \\
= \quad&  (E_A \odot E_A)(X,Y)
\end{align*}
and the counit $\varepsilon : E_A \to I$ simply composes the two morphisms. Because $\Pastro_\M$ is oplax monoidal, the Tambara module $\Pastro_\M E_A$ has an induced comonoid structure.

\begin{lemma}
For an object $X$ in a monoidal category $\C$, a comonoid structure $(X,\Delta,\varepsilon)$ is equivalent to a lax monoidal structure on $\C(X, -)$.
\end{lemma}
\begin{proof}
\todo{not hard}
\end{proof}

Therefore the functor $\Tamb_\C(\M)(\Pastro_\M E_A, -)$ is canonically lax monoidal, and so the isomorphic functor $(U-)(A,A)$ is also.

\begin{proposition}
If $p : S \hto A$ is an optic then the associated natural transformation $(U-)(A,A) \to (U-)(S,S)$ is a lax monoidal.
\end{proposition}
\begin{proof}
The final few isomorphisms used in Proposition \ref{prop-profunctor-optics-are-optics} are essentially window dressing. Suppose we have an element $\langle l, e_1, e_2, r \rangle$ of 
\begin{align*}
(U\Pastro_\M E_A)(S,S) = \int^{M \in \M} \int^{C,D \in \C} \C(S, MC) \times E_A(C,D) \times \C(MD,S).
\end{align*}
The round trip through the isomorphism $\Phi : (U\Pastro_\M E_A)(S,S) \cong \PreOptic_\M((S, S), (A, A))$ demonstrates that $\langle l, e_1, e_2, r \rangle = \langle (M e_1)l, \id_A, \id_A, r(M e_2)\rangle$. Hence we may always assume we have a representative of the form $\langle l, \id_A, \id_A, r \rangle$. This representative corresponds to an optic via the isomorphism $\Phi$ if both $rl = \id_S$ and $\langle lr\rangle = \langle \phi A \rangle$ in $\int^{M \in \M} \C(M A, M A)$. Note that just as we can ``strictify'' a representative of an optic, if we have an element of $(U\Pastro_\M E_A)(S,S)$ that corresponds to an optic we may always choose a representative so that the latter identity holds on the nose.

Next we examine the image of the optics through the Yoneda isomorphism \[ (U\Pastro_\M E_A)(S,S) \cong [\Tamb_\C(\M), \Set](\Tamb_\C(\M)(\Pastro_\M E_A, -), (U-)(S,S)) \] 
By the definition of the Yoneda isomorphism, a natural transformation \[ p : \Tamb_\C(\M)(\Pastro_\M E_A, -) \to (U-)(S,S) \] will correspond to an optic iff the image of the identity morphism $\id : \Pastro_\M E_A \to \Pastro_\M E_A$ through the component $p_{\Pastro_\M E_A}$ is an optic in $(U\Pastro_\M E_A)(S, S)$. 

Finally the isomorphism \[ [\Tamb_\C(\M), \Set](\Tamb_\C(\M)(\Pastro_\M E_A, -), (U-)(S,S)) \cong \; [\Tamb_\C(\M), \Set]((U-)(A,A),(U-)(S,S)) \] is given by precomposition by the isomorphism of Lemma \ref{lemma-rep}, so in summary, a ``profunctor preoptic'' $p : (U-)(A, A) \to (U-)(S,S)$ is a lawful optic iff $p_{\Pastro_\M E_A}(\langle \id_A, \id_A, \id_A, \id_A \rangle)$ is an optic in $(U\Pastro_\M E_A)(S, S)$. 

To show $p$ is monoidal, we must show that both diagrams
\[
\begin{tikzcd}
UP(A,A) \times UQ(A,A) \ar[r, "p_P \times p_Q"] \ar[d, swap] & UP(S, S) \times UQ(S, S)  \ar[d] \\
U(P \odot Q)(A,A)  \ar[r, "p_{P \odot Q}", swap] & U(P \odot Q)(S,S)
\end{tikzcd}
\]
and
\[
\begin{tikzcd}
1 \ar[d] \ar[dr] & \\
UI(A,A)  \ar[r, "p_I", swap] & UI(S,S)
\end{tikzcd}
\]
commute. For the first we have
\begin{align*}
\langle (p_P \times p_Q)(x, y) \rangle &= \langle (UP)(l,r) \alpha_{A,A,M} x,  (UQ)(l,r) \beta_{A,A,M} y\rangle \\
&= \langle (UP)(l,lr) \alpha_{A,A,M} x, (UQ)(\id_{MA},r) \beta_{A,A,M} y\rangle && \text{(by the coend relations)} \\
&= \langle (UP)(l,\phi A) \alpha_{A,A,M} x, (UQ)(\id_{MA},r) \beta_{A,A,M} y\rangle && \text{(as $lr = \phi A$)} \\
&= \langle (UP)((\phi A)l,\id_{MA}) \alpha_{A,A,M} x, (UQ)(\id_{MA},r) \beta_{A,A,M} y\rangle && \text{(by the coend relations again)} \\
&= \langle (UP)(lrl,\id_{MA}) \alpha_{A,A,M} x, (UQ)(\id_{MA},r) \beta_{A,A,M} y\rangle && \text{(as $lr = \phi A$ again)} \\
&= \langle (UP)(l,\id_{MA}) \alpha_{A,A,M} x, (UQ)(\id_{MA},r) \beta_{A,A,M} y\rangle && \text{(as $rl = \id_S$)} \\
&= (U(P \odot Q))(l,r) \langle \alpha_{A,A,M} x, \beta_{A,A,M} y \rangle && \text{(by definition of $P \odot Q$)} \\
&= (U(P \odot Q))(l,r) (\alpha_{A,A,M} \odot \beta_{A,A,M}) \langle x, y \rangle && \text{(by definition of $\alpha \odot \beta$)} \\
&= p_{P \odot Q} \langle x, y \rangle
\end{align*}
And for the second:
\begin{align*}
p_I(\id_A) = (UI)(l,r) (M \id_A) = r \id_{MA} l = rl = \id_S
\end{align*}
as required.
\end{proof}

\todo{I don't see why the converse should be true in general, but it seems to be true in specific situations:}

\begin{proposition}
Let $\C = \Set$ and suppose $p : S \hto A$ is a pre-lens such that the associated natural transformation $p : (U-)(A,A) \to (U-)(S,S)$ is lax monoidal. Then $p$ is a lens.
\end{proposition}
\begin{proof}
\todo{Show that it satisfies the getput, putget and putput laws. This strategy works only with $\times$ and in $\Set$.}
\end{proof}

\todo{
\section{Degenerate Optics}
The Haskell lens library has some other `optics' that aren't proper optics, but still integrate nicely with the rest of the ecosystem, e.g. Getters, Reviews and Folds. I am starting to doubt those fit into this framework though.}

\section{Optic Families}


\section{Indexed Optics}
\todo{
This possibly looks like
\begin{align*}
\int^{M \in \M} \C(S, M(IA)) \times \C(M A', S')
\end{align*}
with another monoidal action $I$. The $I$ would need to distribute over $M$ in some way. ``Coindexed'' optics seem to have 
\begin{align*}
\int^{M \in \M} \C(S, J(MA)) \times \C(M A', S')
\end{align*}
which is why they pass information `in the other direction' but still transform covariantly.
}
\section{Merely Well-Behaved Optics}
\todo{
The lenses that the UPenn crew look at are less restricted. They obey the $\fput\fget$ and $\fget\fput$ laws but not the $\fput\fput$ law. 
}

\begin{example}
The following example is taken from \cite{AClearPictureOfLensLaws}. The ``change counter'' lens $\bN \times A \hto A$ for a set $A$ has $\fput$ and $\fget$ given by:
\begin{align*}
\fget(n, a) &= a \\
\fput((n, a), a') &= \begin{cases}
(n, a) & \text{if } a = a' \\
(n+1, a') & \text{otherwise}
\end{cases}
\end{align*}
This example is typical of (merely) well-behaved lenses: there is metadata carried along with the target of a lens that mutates as the lens is used.
\end{example}

\begin{example}
A minimal example is the following:
\begin{align*}
\fget : \{A,B,C\} &\to \{X, Y\} \\
A &\mapsto X \\
B &\mapsto X \\
C &\mapsto Y 
\end{align*}
\begin{align*}
\fput : \{A,B,C\} \times \{X, Y\} &\to \{A,B,C\} \\
(A,X) &\mapsto A \\
(B,X) &\mapsto B \\
(C,X) &\mapsto B \\
(A,Y) &\mapsto C \\
(B,Y) &\mapsto C \\
(C,Y) &\mapsto C
\end{align*}
\todo{So here the $\fput\fget$ composite is
\begin{align*}
(A,X) &\mapsto (A,X) \\
(B,X) &\mapsto (B,X) \\
(C,X) &\mapsto (B,X) \\
(A,Y) &\mapsto (C,Y) \\
(B,Y) &\mapsto (C,Y) \\
(C,Y) &\mapsto (C,Y)
\end{align*}
}
\end{example}

\todo{The condition on $\langle l, r \rangle$ seems to be that $rl = \id_S$ and $\pi_2lr = \pi_2$. I can't see how to generalise this second condition to optics other than lenses.}

\section{The Challenge of Effectful Optics}

Past attempts \cite{ReflectionsOnMonadicLenses} to define effectful lenses have wrapped the result of one or both of $\fget$ and $\fput$ with a monad. This approach is a little ad-hoc; it is not obvious what the laws ought to be and there is no clear generalisation to other varieties of optic. The definition of optic given in this paper suggests we instead set $\C$ to be the Kleisli category for some monad.

\section{Lenses in Linear Types}

\section{A Double-Categorical Perspective?}

Dual pairs for the functors $(- \times A) : \M \to \C$ give a monoid in $PProf$.


\section{Conclusion}

\bibliographystyle{alpha}
\bibliography{optics.bib}
\end{document}