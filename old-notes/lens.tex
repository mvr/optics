\documentclass[11pt,a4paper]{amsart}
\usepackage[utf8]{inputenc}
\usepackage{amsthm}
\usepackage{amsfonts}
\usepackage{amsmath}
\usepackage{amssymb}
\let\amssquare\square
\usepackage{mathtools}
\usepackage{stmaryrd}
\usepackage{tensor}
\usepackage[mathscr]{eucal}
\usepackage{marvosym}
\usepackage[left=2cm,right=2cm,top=2cm,bottom=2cm]{geometry}

\usepackage{tikz-cd}
\usetikzlibrary{cd}
\usetikzlibrary{arrows}

\theoremstyle{plain}
\newtheorem{theorem}{Theorem}[section]
\newtheorem{axiom}[theorem]{Axiom}
\newtheorem*{theoremstar}{Theorem}
\newtheorem{fact}[theorem]{Fact}
\newtheorem{proposition}[theorem]{Proposition}
\newtheorem{lemma}[theorem]{Lemma}
\newtheorem{corollary}[theorem]{Corollary}

\theoremstyle{definition}
\newtheorem{definition}[theorem]{Definition}
\newtheorem{convention}[theorem]{Convention}
\newtheorem{construction}[theorem]{Construction}
\newtheorem{example}[theorem]{Example}
\newtheorem{examples}[theorem]{Examples}
\newtheorem{notation}[theorem]{Notation}
\newtheorem{remark}[theorem]{Remark}
\newtheorem{idea}[theorem]{Idea}
\newtheorem{question}[theorem]{Question}

\newcommand{\id}{\mathrm{id}}

\newcommand{\C}{\mathscr{C}}
\newcommand{\homC}{\underline{\C}}
\newcommand{\D}{\mathscr{D}}
\newcommand{\E}{\mathscr{E}}
\newcommand{\M}{\mathscr{M}}
\newcommand{\N}{\mathscr{N}}

\newcommand{\Pastro}{\mathrm{Pastro}}

\newcommand{\Set}{\mathbf{Set}}
\newcommand{\Cat}{\mathbf{Cat}}
\newcommand{\Prof}{\mathbf{Prof}}
\newcommand{\MonCat}{\mathbf{MonCat}}
\newcommand{\LaxMonCat}{\mathbf{LaxMonCat}}

\newcommand{\Act}{\mathbf{Act}}
\newcommand{\Lens}{\mathbf{Lens}}
\newcommand{\Hask}{\mathbf{Hask}}
\newcommand{\Endo}{\mathbf{Endo}}
\newcommand{\Strong}{\mathbf{Strong}}
\newcommand{\Tamb}{\mathbf{Tamb}}
\newcommand{\Point}{\mathbf{Point}}
\newcommand{\CoPoint}{\mathbf{CoPoint}}
\newcommand{\Traversable}{\mathbf{Traversable}}

\newcommand{\op}{\mathrm{op}}
\newcommand{\ob}{\mathrm{ob}\,}

\newcommand{\fget}{\mathrm{get}}
\newcommand{\fput}{\mathrm{put}}
\newcommand{\fmodify}{\mathrm{modify}}

\newcommand{\todo}[1]{\textcolor{red}{\small #1}}

\newcommand{\isoto}{\xrightarrow{
   \,\smash{\raisebox{-0.65ex}{\ensuremath{\scriptstyle\sim}}}\,}}

\title{Profunctor Lenses}
\author{Mitchell Riley}
\begin{document}
\maketitle

\section{Introduction}

A good introduction would include:

Motivation for Lenses

Motivation for van Laarhoven Lenses

Motivation for Profunctor Lenses

\subsection{Laws}
Whatever formulation of standard lenses is used, there are some laws that are expected to hold. When we work in $\Set$ (or $\Hask$), the laws are typically given as follows.

Given:
\begin{align*}
\fget &: \Lens(a, b) \times a \to b \\
\fput &: \Lens(a, b) \times b \times a \to a
\end{align*}
we have that
\begin{align*}
\fget(l, \fput(l, b, a))   &= b \\
\fput(l, \fget(l, a), a)   &= a \\
\fput(l, b', \fput(l, b, a)) &= \fput(l, b', a)
\end{align*}

To express these laws in a general category with products, however, we must rewrite these in a ``pointfree'' style.

Given:
\begin{align*}
\fget &: \Lens(a, b) \to \C(a, b) \\
\fmodify &: \Lens(a, b) \times \C(b, b) \to \C(a, a)
\end{align*}
we have that
\begin{align*}
\fget(l) \circ \fmodify(l, f) &= f \circ \fget(l) \\
\fmodify(l, \id_b) &= \id_a \\
\fmodify(l, f) \circ \fmodify(l, g) &= \fmodify(l, f \circ g)
\end{align*}
We will refer to these as the get/modify laws.

Similarly, the laws for prisms are typically stated as
\todo{todo}

\subsection{Notation}

We use $\homC(a,b)$ to denote the internal hom of a cartesian closed category $\C$.

\section{Lenses}

\subsection{Monoidal actions}

\begin{definition}
An \emph{action} of a monoidal category $(\M, \otimes, 1)$ on a category $\C$ is a functor $\cdot : \M \times \C \to \C$ such that there are natural isomorphisms $m \cdot (n \cdot x) \isoto (m \otimes n) \cdot x$ and $1 \cdot x \isoto x$ for any $m, n \in \M$ and $x \in \C$. These isomorphism must satisfy the expected associativity and unit axioms. We will leave these isomorphisms implicit to reduce the amount of noise in our diagram. \todo{draw diagrams?}

Equivalently, an action of a monoidal category is a strong monoidal functor $\M \to [\C, \C]$, where $[\C, \C]$ is equipped with the monoidal structure $([\C, \C], \circ, \id_\C)$.
\end{definition}

Examples abound: 
\begin{itemize} 
\item Any monoidal category acts on itself via left multiplication. 
\item Any subcategory $\M \subset [\C, \C]$ that is closed under composition is a monoidal category, which acts on $\C$ simply by functor application.
\item If $\C$ is equipped with an action of $\M$, then any functor category $[\D, \C]$ has an action of $\M$ computed pointwise.
\end{itemize}

We will also be interested in actions that are \emph{not} functorial in $\C$:

\begin{definition}
A \emph{dodgy monoidal action} of $\M$ on $\C$ is a functor $\cdot : \M \times \ob \C \to \C$, such that the same conditions hold as above. Here, $\ob \C$ is the category with the same objects as $\C$ but only identity morphisms.
\end{definition}

Most of what follows is still true for dodgy monoidal actions. \todo{But I am yet to check carefully which parts fail}

\subsection{Lenses}

\begin{definition}
Let $\C$ be a category equipped with an action of the monoidal category $(\M, \otimes, 1)$. The \emph{category of lenses $\Lens_\C(\M)$} has objects the objects of $\C$, and morphisms $a$ to $b$ given by pairs $(m \in \M, a \isoto m \cdot b)$, where the second component is an isomorphism. However, we identify lenses that can be connected by an isomorphism in $\M$: $(m \in \M, l : a \isoto m \cdot b)$ is identified with $(m' \in \M, l' : a \isoto m' \cdot b)$ iff there exists an isomorphism $\alpha : m \isoto m'$ such that $l' = (\alpha\cdot b)l$. To save on notation, we will simply write $l : a \isoto m \cdot b$ for a representative of a lens $l$.

The identity lens is simply $\id_a : a \isoto 1 \cdot a$, and the composition of two lenses $a \isoto m \cdot b$ and $b \isoto n \cdot c$ is given by
\begin{align*}
a \isoto m \cdot b \isoto m \cdot (n \cdot c) \isoto (m \otimes n) \cdot c
\end{align*}
It is not difficult to check that this composition is well defined.
\end{definition}
\todo{TODO: Consider flipping the direction of a lens to avoid op later?}

We may consider the slice category $\Act(\C) := \MonCat/[\C, \C]$ of monoidal actions on $\C$, where $\MonCat$ is the category of monoidal categories and strong monoidal functors. For any map $F : \M \to \N$ in $\Act(\C)$, we have a functor $\Lens_\C(\M) \to \Lens_\C(\N)$ which sends any lens $a \isoto m \cdot b$ to $a \isoto Fm \cdot b$. Here, the map is identical: $m \cdot b = Fm \cdot b$.  In this way, $\Lens_\C$ forms a functor $\Act(\C) \to \Cat$.

Unfortunately, the above definition of a lens is difficult to work with, both categoricaly and programmatically. Instead, we embed $\Lens_\C(\M)(a, b)$ inside $\int^{m \in M} \C(a, m\cdot b) \times \C(m\cdot b, a)$, and work with the latter.

\begin{lemma}
The canonical map $I : \Lens_\C(\M)(a, b) \to \int^{m \in M} \C(a, m\cdot b) \times \C(m\cdot b, a)$ that sends $l : a \isoto m \cdot b$ to $[(l, l^{-1})]$ is injective.
\end{lemma}
\begin{proof}

\end{proof}

We now wish to characterise the elements of $\int^{m \in M} \C(a, m\cdot b) \times \C(m\cdot b, a)$ that correspond to lenses.

\begin{proposition}\label{characterise-lenses-in-coend}
An element $[(l, r)] \in \int^{m \in M} \C(a, m\cdot b) \times \C(m\cdot b, a)$ corresponds to a lens iff $rl = \id_A$ and $[lr] = [\id_{m\cdot b}]$ as elements of $\int^{m\in \M} \C(m \cdot b, m \cdot b)$, for some $m \in \M$.
\end{proposition}
\begin{proof}
If an element corresponds to a lens, there is a representative $(l, r)$ such that $rl = \id_a$ and $lr = \id_{m\cdot b}$, so certainly $[lr] = [\id_{m\cdot b}]$ in $\int^{m\in \M} \C(m \cdot b, m \cdot b)$.

\todo{This is wrong} Conversely, suppose $rl = \id_A$ and $[lr] = [\id_{m\cdot b}]$ in $\int^{m\in \M} \C(m \cdot b, m \cdot b)$. The coend relations imply that there exist $p : m \cdot b \to n \cdot b$ and $q : n \cdot b \to m \cdot b$ such that $plrq = \id_{n\cdot b}$ and $qp = \id_{m\cdot b}$. 

The same relations hold in $\int^{m \in M} \C(a, m\cdot b) \times \C(m\cdot b, a)$, so $[(l, r)] = [(pl, rq)]$. Now $plrq = \id_{n\cdot b}$ on the nose, and $rqpl = rl = \id_a$, so $[(l, r)]$ represents the lens $pl : a \isoto n \cdot b$.
\end{proof}

\section{Examples}

Viewing lenses as a subset of $\int^{m \in M} \C(a, m\cdot b) \times \C(m\cdot b, a)$ allows for some more practical descriptions of lenses for particular choices of $\M$.

\subsection{Isos}

Let $\M = \mathbf{1}$. Then the set of potential lenses between $a, b \in \C$ is 
\begin{align*}
\int^{m \in \mathbf{1}} \C(a, m \cdot b) \times \C(m \cdot b, a)
&\cong \C(a, 1 \cdot b) \times \C(1 \cdot b, a) \\
&\cong \C(a, b) \times \C(b, a)
\end{align*}

\subsection{Lenses}

Suppose $\C$ has finite products, and let $\M = (\C, \times, 1)$, then
\begin{align*}
(\Pastro_\C(\times)E_b)(a,a) 
&\cong \int^{m \in \C} \C(a, m \times b) \times \C(m \times b, a) \\
&\cong \int^{m \in \C} \C(a, m) \times \C(a, b) \times \C(m \times b, a) \\
&\cong  \C(a, b) \times \C(a \times b, a)
\end{align*}
This is the traditional getter/setter form of lenses. 

\begin{proposition}
An element of $\C(a, b) \times \C(a \times b, a)$ corresponds to a lens iff it satisfies the get/modify lens laws.
\end{proposition}
\begin{proof}
\todo{todo}
\end{proof}

\subsection{Prisms}

Suppose $\C$ has finite coproducts, and let $\M = (\C, \sqcup, 0)$, then
\begin{align*}
(\Pastro_\C(\sqcup)E_b)(a,a) 
&\cong \int^{m \in \C} \C(a, m \sqcup b) \times \C(m \sqcup b, a) \\
&\cong \int^{m \in \C} \C(a, m \sqcup b) \times \C(m, a) \times \C(b, a) \\
&\cong \C(a, a \sqcup b) \times \C(b, a)
\end{align*}

\todo{Show that the laws follow from our law}

\subsection{Setters}
To define Setters, we must first define what it means for a functor to be \emph{strong}.

\begin{definition}
\todo{The extra generality may not be necessary}
Suppose $\C$ is equipped with an action of $\M$. A \emph{strength} for a functor $F : \C \to \C$ is a natural transformation
\begin{align*}
s_{m,a} : m \cdot F a \to F(m \cdot a)
\end{align*}
such that both
\[
\begin{tikzcd}
m \cdot n \cdot Fa \ar[r, "s_{n,a}"] \ar[dr, "s_{m\otimes n, a}" below left]  & m \cdot F(n\cdot a) \ar[d, "s_{m, n\cdot a}" right] \\
& F(m \cdot n \cdot a)
\end{tikzcd}
\]
and
\[
\begin{tikzcd}
1 \cdot F a \ar[r, "s_{1,a}"] \ar[dr, "\cong" below left]  & F(1 \cdot a) \ar[d, "\cong" right] \\
& F a
\end{tikzcd}
\]
commute.
\end{definition}

\begin{definition}
There is a category $\Strong_\C(\M)$ of functors equipped with strengths, where the morphisms are natural transformations that respect the extra structure. There is an evident forgetful functor $U : \Strong_\C(\M) \to [\C, \C]$.
\end{definition}

The composition of two strong functors itself has a canonical strength, given by the composite
\begin{align*}
m \cdot FG a \xrightarrow{s_{m, Ga}} F(m \cdot G a) \xrightarrow{Fs_{m,a}} FG (m \cdot a)
\end{align*}

In this way, $\Strong_\C(\M)$ is also a monoidal category under composition. This category also acts on $\C$, via the action of the underlying functors.

Now, returning to Setters, suppose $\C$ is cartesian closed, and let $\M = \Strong_\C(\times)$. First, a lemma:

\begin{lemma}
$\C(a, Fb) \cong \Strong_\C(\times)(a \times \homC(b, -), F)$
\end{lemma}
\begin{proof}
Given a $f \in \C(a, Fb)$, let $\eta$ be the natural transformation with components
\begin{align*}
a \times \homC(b, c) \xrightarrow{f \times \id} Fb \times \homC(b, c) \xrightarrow{s_{\homC(b, c), b}}  F(b \times \homC(b, c)) \xrightarrow{F\epsilon} Fc
\end{align*}
where $s$ is the strength for $F$ and $\epsilon$ is the evaluation map.

Conversely, given an $\eta$, the component at $b$ is a function $g : a \times \homC(b, b) \to Fb$; we recover $f$ as $f(a) = g(a, \id_b)$.

Again, it is routine to check these are inverses.
\end{proof}

This is a generalisation of Proposition 2.2 in \cite{MR3482285}, which only considered functors $\Set \to \Set$ (all of which are strong).

Now, the set of (possibly invalid) lenses with respect to this action is:
\begin{align*}
\int^{F \in \Strong_\C(\times)} \C(a, F b) \times \C(F b, a) 
&\cong \int^{F} \Strong_\C(\times)(a \times \homC(b, -), F) \times \C(F b, a) \\ 
&\cong \C(a \times \homC(b, b), a) \\
&\cong \C(\homC(b, b), \homC(a, a))
\end{align*}

\subsection{Views and Reviews}

Here we make use of dodgy monoidal actions. Let $\M=\Point(\C)$ be the category of \emph{pointed assignments}: functors $
F : \ob C \to C$ equipped with natural transformations $\epsilon : 1 \to F$.

As in earlier sections, we must charactarise $\C(a, Fb)$ as the set of natural transformation from some assignment to $F$. Let $C_{a,b}$ be the pointed assignment such that $C_{a,b}(b) = a \sqcup b$ and $C_{a,b}(c) = c$ for all $c \neq b$.

\begin{lemma}
$\C(a, Fb) \cong \Point(\C)(C_{a,b}, F)$
\end{lemma}
\begin{proof}
\todo{boring}
\end{proof}

Then:

\begin{align*}
\int^{F \in \Point(\C)} \C(a, F b) \times \C(F b, a) 
&\cong \int^{F} \Point(\C)(C_{a,b}, F) \times \C(F b, a) \\ 
&\cong \C(a \sqcup b, a) \\
&\cong \C(a, a) \times \C(b, a)
\end{align*}



\section{Profunctors and Tambara Modules}

\begin{definition}
A \emph{profunctor} between $\C$ and $\D$ is a functor $\C^\op \times \D \to \Set$. We have a category $\Prof(\C, \D) := [\C^\op \times \D, \Set]$.
\end{definition}

\begin{definition}
For any $a \in \C$, define the \emph{exchange} profunctor to be $E_a := \C(-,a) \times \C(a,=)$. 
%\todo{For lack of better names} We also define the \emph{lefty} and \emph{righty} profunctors to be $L_b := \C(-, b)$ and $R_b := \C(b, =)$ respectively.
\end{definition}

\begin{definition}
Given profunctors $P \in \Prof(\C, \D)$ and $Q \in \Prof(\D,\E)$, the composite $Q \star P \in \Prof(\C, \E)$ is given by
\begin{align*}
(Q \star P)(c,e) = \int^{d \in \D} P(c,d) \times Q(d,e)
\end{align*}
In particular, for any $\C$, $\Prof(\C,\C)$ is a monoidal category with respect to $\star$. The identity for this composition is given by $\C(-,=)$. Details and proofs can be found in \cite{loregian2015co}.
\end{definition}

Any functor $F : \C \to \D$ can be turned into a profunctor in two different ways: $F_* \in \Prof(\D, \C)$ given by $F_* = \D(-,F=)$, and  $F^* \in \Prof(\C, \D)$ given by $F^* = \D(F-,=)$.

\begin{lemma}\label{procomp-downstar-is-comp}
Let $F : \C \to \D$ and $G : \D \to \E$, so that $F_* \in \Prof(\D, \C)$ and $G_* \in \Prof(\E,\D)$. Then $F_* \star G_* \cong (GF)_*$
\end{lemma}
\begin{proof}
\todo{todo}
\end{proof}

%It is also easily verified that $L_b \star R_b \cong E_b$, and $R_b \star L_b$ is the constant profunctor at $\C(b,b)$.

The following definition is a generalisation of an idea that appears in \cite{MR2425558}.

\begin{definition}
Suppose $\C$ is equipped with an action of $\M$, and let $P \in \Prof(\C, \C)$ be a profunctor. A \emph{Tambara module structure} for $P$ with respect to $\M$ is a natural family of maps:
\begin{align*}
\alpha_{a,b,m} : P(a, b) \to P(m \cdot a, m \cdot b)
\end{align*}
such that $\alpha_{a,b,1} = \id_{P(a,b)}$, and the following diagram commutes:
\[
\begin{tikzcd}
P(a,b) \ar[r, "\alpha_{a,b,m}"] \ar[dr, "\alpha_{m\otimes n, a, b}" below left] & P(m \cdot a, m \cdot b) \ar[d, "\alpha_{n, m \cdot a, m \cdot b}" right] \\
& P(n \cdot (m \cdot a), n \cdot (m \cdot b))
\end{tikzcd}
\]
\end{definition}

The identity profunctor $\C(-,=)$ has a module structure with respect to any action, as every action is functorial.

If $P \in \Prof(\C, \D)$ and $Q \in \Prof(\D,\E)$ are equipped with module structures $\alpha$ and $\beta$ respectively, there is a canonical module structure on $Q \star P$ given by
\begin{align*}
(Q \star P)(c,e) &= \int^{d \in D} P(c,d) \times Q(d,e) \\ 
&\to \int^{d \in D} P(m \cdot c, m\cdot d) \times Q(m \cdot d, m \cdot e) \\
&\to \int^{d \in D} P(m \cdot c, d) \times Q(d, m \cdot e) \\
&= (Q \star P)(m \cdot c, m \cdot e)
\end{align*}
Here, the second map is induced by the universal property of the coend: Each $P(m \cdot c, m\cdot d) \times Q(m \cdot d, m \cdot e)$ has a canonical inclusion into $\int^{d \in D} P(m \cdot c, d) \times Q(d, m \cdot e)$.

\begin{definition}
There is a category $\Tamb_\C(\M)$ of Tambara modules and natural transformations that respect the extra structure, in the sense that the diagram
\[
\begin{tikzcd}
P(a,b) \ar[r, "\alpha_{a,b,m}"] \ar[d, "l_{a,b}" left] & P(m \cdot a, m \cdot b) \ar[d, "l_{m\cdot a,m\cdot b}" right] \\
Q(a,b) \ar[r, "\alpha_{a,b,m}"] & Q(m \cdot a, m \cdot b)
\end{tikzcd}
\]
commutes.
\end{definition}

This category is monoidal with respect to $\star$ as given above. There is an evident forgetful functor $U : \Tamb_\C(\M) \to \Prof(\C, \C)$ that is strong monoidal.

\begin{definition}
Let $\mathrm{Pastro}_\C(\M) : \Prof(\C, \C) \to \Tamb_\C(\M)$ be the functor
\begin{align*}
\mathrm{Pastro}_\C(\M)(P) := \int^{m \in \M} \C(m \cdot -, =) \star P \star \C(-, m \cdot =)
\end{align*}
Or, in other words, 
\begin{align*}
\mathrm{Pastro}_\C(\M)(P)(a,b) := \int^{m \in \M} \int^{c,d \in \C} \C(a, m \cdot c) \times P(c,d) \times  \C(m \cdot d, b)
\end{align*}
The module structure $\alpha_{a,b,m} : \mathrm{Pastro}_\C(\M)(P)(a, b) \to \mathrm{Pastro}_\C(\M)(P)(m \cdot a, m \cdot b)
$ is given by the composite
\begin{align*}
\mathrm{Pastro}_\C(\M)(P)(a,b) &= \int^{n \in \M} \int^{c,d \in \C} \C(a, n \cdot c) \times P(c,d) \times  \C(n \cdot d, b) \\
&\to \int^{n \in \M} \int^{c,d \in \C} \C(m \cdot a, m \cdot (n \cdot c)) \times P(c,d) \times  \C(m \cdot (n \cdot d), m \cdot b) \\
&\cong \int^{n \in \M} \int^{c,d \in \C} \C(m \cdot a, (m \otimes n) \cdot c) \times P(c,d) \times  \C((m \otimes n) \cdot d, m \cdot b) \\
&\to \int^{n \in \M} \int^{c,d \in \C} \C(m \cdot a, n \cdot c) \times P(c,d) \times  \C(n \cdot d, m \cdot b) \\
&= \mathrm{Pastro}_\C(\M)(P)(m \cdot a, m \cdot b)
\end{align*}
where the final map is again induced by the universal property of the coend. \todo{This does not work if the action is not functorial in $\C$!!!}
\end{definition}

\begin{proposition}
$\Pastro_\C(\M)$ is left adjoint to $U$.
\end{proposition}
\begin{proof}
We prove this directly by providing natural inverse maps
\begin{align*}
\alpha : \Tamb_\C(\M)(\Pastro_\C(\M)(P), Q) \cong \Prof(\C, \C)(P, UQ) : \beta
\end{align*}
Given a natural transformation $f : \Pastro_\C(\M)(P) \to Q$, let $\alpha(f) : P \to UQ$ be the natural transformation with components
\begin{align*}
P(a,b) \xrightarrow{(id, -, id)} \C(a, 1 \cdot a) \times P(a,b) \times \C(1 \cdot b, b) \xrightarrow{f} Q(a,b)
\end{align*} 
Given a $g : P \to UQ$, let $\beta(g) : \Pastro_\C(\M)(P) \to Q$ be the natural transformation with components
\begin{align*}
\Pastro_\C(\M)(P)(a,b) &= \int^{m \in \M} \int^{c,d \in \C} \C(a, m \cdot c) \times P(c,d) \times  \C(m \cdot d, b) \\
&\xrightarrow{g}
\int^{m \in \M} \int^{c,d \in \C} \C(a, m \cdot c) \times Q(c,d) \times  \C(m \cdot d, b) \\
&\xrightarrow{\alpha}
\int^{m \in \M} \int^{c,d \in \C} \C(a, m \cdot c) \times Q(m\cdot c,m\cdot d) \times  \C(m \cdot d, b) \\
&\to Q(a,b) \\
\end{align*}
\todo{Prove that $\beta(g)$ preserves the module structure}\\
It is routine to check that these $\alpha$ and $\beta$ are inverses.
\end{proof}



\section{Profunctor Lenses}

For any $a \in \C$, we have a functor $-(a, a) : \Tamb_\C(\M) \to \Set$. Note that this functor is lax monoidal: there is a canonical inclusion \todo{flip P and Q?}
\begin{align*}
i_{P,Q} : P(a,a)\times Q(a,a) \to \int^{d \in D} Q(a,d) \times P(d,a) = (P \star Q)(a,a)
\end{align*}
and function $i : 1 \to \C(a,a)$ that selects $\id_a$.

\begin{lemma}
The functor $-(a, a) : \Tamb_\C(\M) \to \Set$ is representable: there is a natural isomorphism
$-(a, a) \cong \Tamb_\C(\M)(\Pastro_\C(\M)E_b, -)$
\end{lemma}
\begin{proof}
We have the chain of natural isomorphisms:
\begin{align*}
&-(a,a) \\
&\cong \int_{c,d \in \C} \Set(\C(a, c) \times \C(d, a), -(c,d)) \\
&\cong \int_{c,d \in \C} \Set(E_a(c,d), -(c,d)) \\
&\cong \Prof(E_a, U-) \\
&\cong \Tamb_\C(\M)(\Pastro_\C(\M)E_a, -)
\end{align*}
\end{proof}

To any lens $l = a \isoto m \cdot b$ we associate a natural transformation $l' : -(b, b) \to -(a, a)$ with components 
\begin{align*}
P(b, b) \xrightarrow{\alpha_{b,b,m}} P(m\cdot b, m\cdot b) \xrightarrow{P(l, l^{-1})} P(a,a)
\end{align*}
where $\alpha$ is the module structure on $P$.

Given such a natural transformation $-(b, b) \to -(a, a)$, we would like to recover an object $m \in \M$ and maps $a \to m\cdot b$, $m \cdot b \to a$. 

\begin{lemma}
\begin{align*}
[\Tamb_\C(\M), \Set](-(b, b),-(a, a)) &\cong \int^{m \in \M} \C(a, m \cdot b) \times \C(m \cdot b, a)
\end{align*}
\label{pastro-exchange-characterisation}
\end{lemma}
\begin{proof}
We have the chain of isomorphisms:
\begin{align*}
&[\Tamb_\C(\M), \Set](-(b, b),-(a, a)) \\
&\cong [\Tamb_\C(\M), \Set](\Tamb_\C(\M)(\Pastro_\C(\M)E_b, -), -(a, a)) \\
&\cong (\Pastro_\C(\M)E_b)(a,a) \\
&= \int^{m \in \M} \int^{c,d \in \C} \C(a, m \cdot c) \times E_b(c,d) \times \C(m \cdot d, a) \\
&\cong \int^{m \in \M} \int^{c,d \in \C} \C(a, m \cdot c) \times \C(c,b) \times \C(b, d) \times \C(m \cdot d, a) \\
&\cong \int^{m \in \M} \C(a, m \cdot b) \times \C(m \cdot b, a)
\end{align*}
\end{proof}

\begin{corollary}
A natural transformation $l : -(b, b) \to -(a, a)$ is determined by the value of the component of $l$ at $\Pastro_\C(\M)E_b$, on the canonical element $(\id_b, (\id_b,\id_b),\id_b) \in (\Pastro_\C(\M)E_b)(b, b)$.
\end{corollary}
\begin{proof}
This is exhibited by the action of the intermediate isomorphism
\begin{align*}
[\Tamb_\C(\M), \Set](-(b, b),-(a, a))
\cong (\Pastro_\C(\M)E_b)(a,a)
\end{align*}
\end{proof}

We can now characterise the subset of natural transformations $-(b, b) \to -(a, a)$ that correspond to lenses.

\begin{proposition}
A natural transformation 
$l : -(b, b) \to -(a, a)$ corresponds to a lens iff $l$ is a lax monoidal transformation with repect to $\star$ on $\Tamb_\C(\M)$ and $\times$ on $\Set$, i.e., for all $P, Q \in \Tamb_\C(\M)$, both 
\[
\begin{tikzcd}
P(b, b) \times Q(b, b) \ar[d, "i"left ] \ar[r, "l_P \times l_Q" above] & P(a, a) \times Q(a, a) \ar[d, "i"] \\
(Q\star P)(b, b) \ar[r, "l_{P \star Q}" below]  & (Q\star P)(a, a)
\end{tikzcd}
\]
and
\[
\begin{tikzcd}
& 1 \ar[dl, "i" above left] \ar[dr, "i"] \\
\C(b,b) \ar[rr, "l" below] & &\C(a,a)
\end{tikzcd}
\]
commute.
\end{proposition}
\begin{proof}
Let $l : a \isoto m \cdot b$ be a representative of a lens such that $l$ is invertible. Then in the following diagram:
\[
\begin{tikzcd}[column sep=huge]
P(b, b) \times Q(b, b) \ar[d, "i" left ] \ar[r, "\alpha \times \alpha" above] & P(m\cdot b, m\cdot b) \times Q(m\cdot b, m\cdot b) \ar[d, "i"] \ar[r, "{P(l, l^{-1})\times Q(l, l^{-1})}" above] & P(a, a) \times Q(a, a) \ar[d, "i"] \\
(Q\star P)(b, b) \ar[r, "\alpha_{m,b,b}" below] & (Q\star P)(m\cdot b, m\cdot b) \ar[r, "{(Q\star P)(l, l^{-1})}" below] & (Q\star P)(a, a)
\end{tikzcd}
\]

The left hand square commutes by the definition of the module structure structure on $Q\star P$. Further factoring the right hand square:
\[
\begin{tikzcd}[column sep=huge]
P(m\cdot b, m\cdot b) \times Q(m\cdot b, m\cdot b) \ar[d, "i"] \ar[r, "{P(l, \id)\times Q(\id, l^{-1})}" above] & P(a, m\cdot b) \times Q(m\cdot b, a) \ar[d, "i"] \ar[r, "{P(\id, l^{-1})\times Q(l, \id)}" above] & P(a, a) \times Q(a, a) \ar[d, "i"] \\
(Q\star P)(m\cdot b, m\cdot b) \ar[r, "{(Q\star P)(l, l^{-1})}" below] & (Q\star P)(a, a) \ar[r, "\id" below] &(Q\star P)(a, a)
\end{tikzcd}
\]
The left side commutes by functoriality, and the right side commutes by the the definition of coend. Therefore the associated natural transformation is lax monoidal as requried.

Also, the required triangle commutes, as $\id_b \in \C(b,b)$ is sent to 
\begin{align*}
(l \circ - \circ l^{-1})\alpha_{b,b,m}(\id_b) = (l \circ - \circ l^{-1})(\id_{m\cdot b}) = l \circ l^{-1} = \id_a
\end{align*}
Therefore, the associated natural transformation is lax monoidal.

Conversely, suppose we have a lax monoidal natural transformation $-(b, b) \to -(a, a)$, and let $(l, r) \in \int^{m \in \M} \C(a, m \cdot b) \times \C(m \cdot b, a)$ be a representative of the corresponding element of the coend. The triangle condition exactly states that $rl = \id_a$.

Let $L_b := \int^{m \in \M} \C(-,m\cdot b)$ and $R_b := \int^{m \in \M} \C(m\cdot b,=)$ be elements of $\Tamb_\C(\M)$ with the evident module structures. Note that $R_b \star L_b$ is the constant profunctor at $\int^{m\in \M} \C(m \cdot b, m \cdot b)$. Specialising the above digram to $P = R_b$ and $Q = L_b$, we have:
\[
\begin{tikzcd}[column sep=huge]
R_b(b, b) \times L_b(b, b) \ar[d, "i" left ] \ar[r, "\alpha_{m,b,b} \times \alpha_{m,b,b}" above] & R_b(m\cdot b, m\cdot b) \times L_b(m\cdot b, m\cdot b) \ar[d, "i"] \ar[r, "{R_b(l, r)\times L_b(l, r)}" above] & R_b(a, a) \times L_b(a, a) \ar[d, "i"] \\
(R_b\star L_b)(b, b) \ar[r, "\alpha_{m,b,b}" above] \ar[d, "\cong" left] & (R_b\star L_b)(m\cdot b, m\cdot b) \ar[r, "{(R_b\star L_b)(l,r)}" above] \ar[d, "\cong" right] & (R_b\star L_b)(a, a) \ar[d, "\cong" right]\\
\int^{m\in \M} \C(m \cdot b, m \cdot b) \ar[r, "\id" below] & \int^{m\in \M} \C(m \cdot b, m \cdot b) \ar[r, "\id" below] & \int^{m\in \M} \C(m \cdot b, m \cdot b)
\end{tikzcd}
\] 

Here, the vertical maps are simply composition of morphisms in $\C$. Inspecting the right-hand rectangle, commutativity of the diagram implies that $[lr] = [\id_{m\cdot b}]$ in $\int^{m\in \M} \C(m \cdot b, m \cdot b)$. 

Therefore, by Proposition \ref{characterise-lenses-in-coend}, $(l,r)$ represents a lens.
\end{proof}


We will refer to the above diagrams as the \emph{lens laws}, and say that natural transformations $-(b, b) \to -(a, a)$ are \emph{valid lenses} if they satisfy it.

\begin{theorem}
The functor $I : \Lens_\C(\M) \to \LaxMonCat(\Tamb_\C(\M), \Set)^\op$ is full and faithful, where $\LaxMonCat(\Tamb_\C(\M), \Set)$ is the category of lax monoidal functors and monoidal natural transformations.
\end{theorem}
\begin{proof}
\todo{TODO}
\end{proof}

\subsection{``Polymorphic'' lenses}

\todo{This is an important todo}





\section{Haskell Lenses}
To be completely precise, the above sections should have been written in the enriched setting. Here we will do the usual slight of hand and treat $\Hask$ as equivalent to $\Set$.

One reason that the van Laarhoven/profunctor description of lenses is so effective in Haskell, is the ease with which one can represent ends and coends.

\todo{demonstrate}

\subsection{Lenses in other categories}

\section{Future Work}

"Symmetric" Lenses



\begin{figure}[b]
\begin{tabular}{ll}
\hline
Haskell Name & Acting Category \\
\hline
Iso & $\mathbf{1}$ \\
Lens & $(\Hask, \times)$ \\
Prism & $(\Hask, \sqcup)$ \\
Traversal & $(\Traversable(\Hask), \circ)$ \\
Setter & $(\Endo(\Hask), \circ)$ \\
Review & $(\Point(\Hask), \circ)$ \\
Getter/View & $(\CoPoint(\Hask), \circ)$ \\
Fold & ??? \\
\hline
\end{tabular}
\caption{Classes of $\Lens$es used in Haskell}
\end{figure}

\begin{figure}[b]
\begin{tikzpicture}[node distance=3.5cm,line width=0.5pt]
\node(1) at (0,0) {$1$};
\node(Lens) [above left of =1] {$- \times D$};
\node(Prism) [above right of =1] {$- \sqcup D$};
\node(Traversal) [above right of =Lens] {$\Traversable(\Hask)$};
\node(Setter) [above of =Traversal] {$\Endo(\Hask)$};
\node(Getter) [above left of =Lens] {$\CoPoint(\Hask)$???};
\node(Fold)   [above of=Getter] {Fold ???};
\node(Review) [above right of =Prism] {$\Point(\Hask)$???};


\draw[right hook->](1) -- (Lens);
\draw[right hook->](1) -- (Prism);
\draw[right hook->](Lens) -- (Traversal);
\draw[right hook->](Prism) -- (Traversal);
\draw[right hook->](Traversal) -- (Fold);
\draw[right hook->](Lens) -- (Getter);
\draw[right hook->](Prism) -- (Review);
\draw[right hook->](Getter) -- (Fold);

\draw[->](Traversal) -- (Review);
\draw[->](Traversal) -- (Setter);


\end{tikzpicture}
\caption{Morphisms between various classes of $\Lens$es. Arrows marked with hooks are fully faithful. \todo{figure out if this matters}
}
\end{figure}

\nocite{*}
\bibliographystyle{amsalpha}
\bibliography{lens}
\end{document}